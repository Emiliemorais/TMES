\chapter{Revisão de Literatura}

Neste capítulo é apresentado um embasamento teórico para o entendimento e justificativa de uso da técnica a ser apresentada no capítulo seguinte.

\section{Conceitos da metodologia ágil}
  Nesta seção são apresentados alguns conceitos utilizados nos métodos ágeis e que são utilizados na técnica a ser apresentada no capítulo seguinte.

  \subsection{\textit{Sprint}}

    Times ágeis trabalham em iterações, que são um período de tempo fixo (\textit{time-boxed}) curto, de não mais que um mês de duração. Neste tempo o time deve entregar um incremento do software. \cite{cohn06} \cite{scrum13}

  \subsection{Histórias de Usuário}

    Uma história de usuário consiste numa representação dos requisitos do software em relação as suas funcionalidades e são 
    expressas geralmente no seguinte formato: \cite{cohn06}

    \begin{centering}
        
        \textit{"Eu como <tipo do usuário>, eu desejo <ação> para <valor de negócio>"}
    
    \end{centering}

  \subsection{\textit{Story Points}}
  
    Em um projeto ágil que utiliza histórias de usuário, pode-se estimar o tamanho de uma história em \textit{story points}.

    \textit{Story points} representam o tamanho geral de uma história, envolvendo o esforço necessário, a complexidade e os
    riscos inerentes para desenvolver a história \cite{cohn06}. O número atribuído à história como seu tamanho não é relevante,
    o que importa é o valor relativo entre as histórias, que é obtido por comparação entre elas \cite{cohn06}.


\section{Abordagens de estimativas em ágeis}

  \citeonline{popli14} apresentam algums tipos de abordagens para estimativas:

  \begin{itemize}

    \item Orientada ao aprendizado: Estimativa baseada na aprendizagem coletiva proveniente de experiências em estimativas anteriores e 
    na opinião de especialistas. 

    \item Baseada na opinião: Nessa abordagem um especialista compara o projeto com projetos que ele conhece.

    \item Métodos de regressão: Nessa abordagem são utilizadas equações para realização das estimativas.

    \item \textit{Bottom-up}: Nessa abordagem cada componente que faz parte do sistema é estimado separadamente e depois essas estimativas são agregadas para formar uma estimativa do todo.

  \end{itemize}

  A utilização dessas abordagens podem ser vistas em alguns métodos bastante difundidos das metodologias ágeis. Um exemplo disso é a atribuição dos \textit{story points} com o uso de técnicas como o \textit{Planning Poker}, nessa técnica é utilizada a abordagem orientada ao aprendizado e a abordagem \textit{Bottom-up}.

  O \textit{Velocity} é usado para medir a taxa de progresso do time, dizendo quantos pontos o time é capaz de desenvolver
  numa \textit{sprint}, podendo ser calculado como a soma dos pontos concluídos durante a iteração \cite{cohn06}. 
  
  Com o tamanho do \textit{software} e o \textit{velocity} do time de desenvolvimento conhecido, fica fácil estimar um prazo para o projeto, dividindo o tamanho pelo \textit{velocity}, obtendo assim a quantidade de \textit{sprints} necessárias, e multiplicando pelo tamanho da \textit{sprint}. \cite{cohn06}

  O cálculo da duração a partir do \textit{Velocity} pode ser considerada como uma estimativa através do método de regressão. Mas a duração
  pode ser calculada também com base na opinião de especialistas utilizando assim a abordagem baseada na opinião.

  Considerando tais aspectos, é possível ver que as abordagens utilizadas dependem bastante da opinião dos envolvidos no projeto.
