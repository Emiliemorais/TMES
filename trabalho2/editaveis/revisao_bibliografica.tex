\chapter{Revisão de Literatura}

  \section{Metodologia ágil}
 
  As metodologias ágeis se baseiam nos valores ágeis publicados no Manifesto Ágil
  (BECK, 2001 \textit{apud} \citeonline{cohn06}), que são:
  
  \begin{itemize}
   \item Indivíduos e interações entre eles mais que processos e ferramentas;
   \item Software em funcionamento mais que documentação abrangente;
   \item Colaboração com o cliente mais que negociação de contratos;
   \item Responder a mudanças mais que seguir um plano.
  \end{itemize}
  
 Seguir uma metodologia ágil em um projeto, também implica em uma abordagem ágil nas estimativas e planejamento, seguindo 
 os valores supracitados \cite{cohn06}. Conceitos importantes são necessários para entender o planejamento e estimativas
 num projeto ágil, como: iteração e \textit{sprints}, \textit{story points}, \textit{velocity}, entre outros.
 
 
  \section{\textit{Sprint}}
  
  Times ágeis trabalham em iterações, que são um período de tempo fixo (\textit{time-boxed}) curto, de não mais que
 um mês de duração \cite{cohn06} \cite{scrum13}. O \textit{Scrum}, um \textit{framework} estrutural ágil, 
 traz o conceito de \textit{Sprint} para o processo de desenvolvimento, que nada mais é que um \textit{container} 
 temporal fixo para outros eventos do \textit{Scrum} \cite{scrum13}. Numa \textit{Sprint} é produzido um incremento 
 do \textit{software}, com base no que foi estabelecido como escopo daquela \textit{Sprint}. Como uma \textit{Sprint} possui
 a duração fixa, mesmo que uma funcionalidade não tenha sido completada nesta \textit{Sprint}, a mesma é encerrada no prazo
 que foi determinado \cite{cohn06}.
 
 De acordo com \citeonline{scrum13}, a 
 \textit{sprint} é composta pela reunião de planejamento da \textit{sprint}, reuniões diárias, trabalho de desenvolvimento,
 revisão da \textit{sprint} e retrospectiva da \textit{sprint}. Os esforços de estimativas se concentram na reunião de
 planejamento da \textit{sprint}, onde são estimadas as histórias de usuários. No começo de cada \textit{sprint}, o time 
 incorpora todo o conhecimento obtido com a \textit{sprint} passada para adaptar e ajustar o planejamento da próxima
 \textit{sprint} \cite{cohn06}.
  
 \section{Estimativa de tamanho da Sprint}