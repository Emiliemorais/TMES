\chapter[Conclusão]{Conclusão}

A partir da realização deste trabalho, proposto para a disciplina de Técnicas de Medição Funcional e Estimativa de Software,
conclui-se que com a utilização do algoritmo CAEA (Constructive Agile Estimation Algorithm) no processo de estimativa de projetos
ágeis, utilizando \textit{Story Points}, obtém-se uma estimativa inicial mais ralista e precisa de custo, tamanho e duração do projeto estimado.

A utilização dos fatores vitais com os diferentes níveis de intensidade fornecem um cenário realista para a estimativa, garantindo uma
aproximação fiel dos números estimados em relação aos números reais. Além disso, está técnica fornece uma base para que profissionais menos
experientes estimem projetos mais precisamente, sem que precisem recorrer à especialistas ou fazerem uso de base de dados históricas para
realizar as astividades de estimativa.

