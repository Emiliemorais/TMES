\chapter[Introdução]{Introdução}

  Atualmente, as metodologias ágeis tem sido bastante difundidas. Com o surgimento concretizado em 2001, através
  da publicação do Manifesto Ágil, trouxe uma série de valores com o intuito de mudar o cenário mundial da produção de Software. Esses valores são: (BECK, 2001 \textit{apud} \citeonline{cohn06})
  
  \begin{itemize}
   \item Indivíduos e interações entre eles mais que processos e ferramentas;
   \item Software em funcionamento mais que documentação abrangente;
   \item Colaboração com o cliente mais que negociação de contratos;
   \item Responder a mudanças mais que seguir um plano.
  \end{itemize}
  
  A adoção da metodologia ágil em um projeto implica em seguir os valores supracitados e dessa forma,
  é necessário adaptar a forma de estimar e de planejar um projeto. 

  Todavia, as estimativas realizadas em ágeis ainda são um pouco obscuras, por se tratar de uma metodologia relativamente nova. Embora, existam alguns conceitos
  bem definidos como o \textit{Velocity}, por exemplo, as estimativas de custo, esforço e prazo ainda são bem relativas. Dessa forma, existe uma gama de estudos na área, o que resulta em diversos métodos diferentes para realizar essas estimativas.

\section{Objetivo do trabalho}

	Considerando os aspectos supracitados esse trabalho tem como objetivo apresentar uma técnica, desenvolvida em um estudo, para estimar tamanho e duração em projetos que utilizam metodologia ágil, considerando aspectos do projeto.

\section{Organização do Trabalho}

	Esse trabalho está organizado em três capítulos: Revisão de Literatura, Técnica proposta e Conclusão.

	\begin{itemize}
		\item No \textit{Capítulo 2. Revisão de Literatura} são apresentados os principais conceitos relacionados à estimativas na metodologia ágil.
    	\item No \textit{Capítulo 3. Técnica proposta} é apresentada o funcionamento e aplicação da técnica pesquisada.
    	\item No \textit{Capítulo 4. Conclusão} são apresentadas as conclusões sobre a técnica pesquisada e do trabalho.
	\end{itemize}


