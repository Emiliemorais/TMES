Como já abordado na introdução deste trabalho, os métodos ágeis ainda não possuem uma técnica ou conceito bem consolidado
para estimativa inicial do tamanho, custo e duração de um projeto de software devido à incerteza associada aos requisitos \cite{bhalerao09}. 
Sabe-se que em abordagens ágeis, a opinião de especialistas e bases de dados históricas de projetos são variáveis 
dependentes para as atividades de estimativa, o que não considera fatores importantes que causam impacto no tamanho, custo e duração 
de um projeto \cite{bhalerao09}. 

Levando em consideração essas afirmações, \citeonline{bhalerao09} perceberam que havia a necessidade de um algoritmo simples que
incorporasse os fatores que afetam o custo, tamanho e duração de projetos e, portanto, criaram o CAEA 
(Constructive Agile Estimation Algorithm) visando fornecer uma técnica mais precisa para uma estimativa mais realista
em métodos ágeis.

