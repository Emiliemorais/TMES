\chapter{Apresentação do Sistema}

  O sistema contado é o REPNBL – Fase 2 – Sistema de Apresentação de Projetos ao Programa Nacional de Banda Larga,
  que é utilizado pelo Ministério das Comunicações.

    \section{Domínio e funções do Software}
	
	O REPNBL é um software que permite a apresentação, gestão da análise e aprovação dos projetos de implantação,
	ampliação ou modernização de redes de telecomunicações que suportem acesso à internet em banda larga, onde Pessoas 
	Jurídicas podem submeter projetos no sistema, que serão recebidos e analisados por funcionários do Ministério das Comunicações.
	
    \section{Usuários do Software}
	
	O software conta basicamente com dois tipos de usuários, os usuários externos e internos ao Ministério das Comunicações.
	Os usuários externos são Pessoas Físicas, que representam Pessoas Jurídicas, que são responsáveis por submeter e manter
	os projetos no sistema. Os usuários internos são funcionários do Ministério, com as devidas permissões, responsáveis por
	receber os projetos submetidos e administrá-los no sistema. 
	
    \section{Contagem}
      
      Esta seção apresenta informações sobre as contagens que foram realizadas.
      
	\subsection{Propósito da contagem}
		
	  Medir o tamanho do pedaço de software, definido pelo escopo da contagem estabelecido, 
	  utilizando três abordagens diferentes, para conhecer o valor do atributo do software, 
	  comparar as medidas obtidas com as diferentes técnicas e desenvolver habilidades na utilização
	  das técnicas de medição utilizadas.

	\subsection{Escopo da contagem}
	  
	  O escopo da contagem definido é composto pelos 8 casos de uso seguintes:
		
	  \begin{itemize}
		  

	      \item UC14 - Ferramentas administrativas;
	      \item UC15 - Enviar mensagens;
	      \item UC16 - Submeter relatório parcial e final;
	      \item UC17 - Analisar relatório;
	      \item UC18 - Visualizar relatório;
	      \item UC19 - Extrair dados;
	      \item UC20 - Visualizar acompanhamento;
	      \item UC21 - Emitir relação de projetos;	

	  \end{itemize}
	  
	  Este mesmo escopo de contagem foi utilizado nas três abordagens de contagem: Pontos de Casos de Uso, Pontos de Função e Pontos de Função COSMIC.
	  
	  \subsubsection{Aspectos das funcionalidades medidas}
	      
	      As funcionalidades do software medido representam um software pronto que está em produção no Ministério das Comunicações,
	      caracterizando as funcionalidades como entregues.
	  
	\subsection{Documentação utilizada}
	    
	    Os seguintes artefatos foram utilizados como fonte de informações para identificar os requisitos funcionais do \textit{software}.
	    
	    \begin{itemize}
	      \item Especificações de Caso de Uso;
	      \item Regras de negócio;
	      \item Proposta de Especificação do Software;
	      \item Protótipos de telas;
	    \end{itemize}