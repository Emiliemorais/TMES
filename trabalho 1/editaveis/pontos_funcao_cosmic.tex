\section{Data da contagem}

  A contagem por Pontos de Função COSMIC foi realizada no dia 27/09/2015.


\section{Nível de granularidade do RFU e nível de decomposição do software}
  
   Como a principal documentação utilizada como base para a contagem foram as especificações de casos de uso, 
   estes foram decompostos em requisitos funcionais menores que representassem todos os seus fluxos.

\vfill
\pagebreak
\section{Processos funcionais}
  
  Esta seção apresenta a contagem realizada para os processos funcionais identificados.
  
  \subsection{Listar projetos}
  
      \begin{table}[!h]
      \centering
      \caption{Processo Funcional - Listar projetos}
      \label{pf_listar_projetos}
      \begin{tabular}{|c|c|c|c|}
      \hline
      \multicolumn{4}{|c|}{Listar projetos}                                                                                                                            \\ \hline
      \textbf{Evento disparador}                                                                                                        & \textbf{Movimentos} & \textbf{Grupo de dados} & \textbf{Pontos} \\ \hline
      \multirow{4}{*}{\begin{tabular}[c]{@{}c@{}}Usuário deseja visualizar os seus projetos.\end{tabular}} & Entry               & Seleciona a visualização   & 1               \\ \cline{2-4} 
																      & Read                & Projeto                 & 1               \\ \cline{2-4} 
																      & Exit                & Projeto    & 1               \\ \hline
    \multicolumn{3}{|c|}{\textbf{TOTAL }}                                                                                                                                     & \textbf{3}               \\ \hline
    \end{tabular}
    \end{table}
  
   \subsection{Cadastrar Portaria de Habilitação}
  
      \begin{table}[!h]
      \centering
      \caption{Processo Funcional - Cadastrar Portaria de Habilitação}
      \label{pf_listar_projetos}
      \begin{tabular}{|c|c|c|c|}
      \hline
      \multicolumn{4}{|c|}{Cadastrar Portaria de Habilitação}                                                                                                                            \\ \hline
      \textbf{Evento disparador}                                                                                                        & \textbf{Movimentos} & \textbf{Grupo de dados} & \textbf{Pontos} \\ \hline
      \multirow{4}{*}{\begin{tabular}[c]{@{}c@{}}Usuário deseja realizar o upload\\ de uma portaria de habilitação, \\assim seleciona um projeto.\end{tabular}} & Entry               & Projeto   & 1               \\ \cline{2-4} 
																      & Write & Projeto(Portaria de Habilitação)                 & 1               \\ \cline{2-4} 
																      & Write & Projeto(Relatório)                 & 1               \\ \cline{2-4} 
																      & Exit                & Projeto(Portaria de Habilitação)    & 1               \\ \cline{2-4}
																      & Exit                & Mensagens               & 1               \\ \hline
    \multicolumn{3}{|c|}{\textbf{TOTAL }}                                                                                                                                     & \textbf{5}               \\ \hline
    \end{tabular}
    \end{table}
  
     \subsection{Submeter Relatório Parcial}
  
      \begin{table}[!h]
      \centering
      \caption{Processo Funcional - Submeter Relatório Parcial}
      \label{pf_submeter_relatorio}
      \begin{tabular}{|c|c|c|c|}
      \hline
      \multicolumn{4}{|c|}{Submeter Relatório Parcial}                                                                                                                            \\ \hline
      \textbf{Evento disparador}                                                                                                        & \textbf{Movimentos} & \textbf{Grupo de dados} & \textbf{Pontos} \\ \hline
      \multirow{4}{*}{\begin{tabular}[c]{@{}c@{}}Usuário deseja enviar um relatório parcial.\end{tabular}} & Entry               & Projeto   & 1               \\ \cline{2-4} 
																      & Read & Projeto(Relatório)                 & 1               \\ \cline{2-4} 
																      & Read & Projeto(Subprojeto)                 & 1               \\ \cline{2-4} 
																      & Write                & Projeto(Relatório)    & 1               \\ \cline{2-4}
																      & Write                & Projeto(Subprojeto)    & 1               \\ \cline{2-4}
																      & Exit                & Mensagens               & 1               \\ \hline
    \multicolumn{3}{|c|}{\textbf{TOTAL }}                                                                                                                                     & \textbf{6}               \\ \hline
    \end{tabular}
    \end{table}
    
      
     \subsection{Visualizar Relatório Parcial}
  
      \begin{table}[!h]
      \centering
      \caption{Processo Funcional - Visualizar Relatório Parcial}
      \label{pf_visualizar_relatorio_parcial}
      \begin{tabular}{|c|c|c|c|}
      \hline
      \multicolumn{4}{|c|}{Visualizar Relatório Parcial}                                                                                                                            \\ \hline
      \textbf{Evento disparador}                                                                                                        & \textbf{Movimentos} & \textbf{Grupo de dados} & \textbf{Pontos} \\ \hline
      \multirow{4}{*}{\begin{tabular}[c]{@{}c@{}}Usuário deseja visualizar um\\ relatório parcial.\end{tabular}} & Entry               & Projeto & 1               \\ \cline{2-4} 
																      & Read & Projeto(Relatório)                 & 1               \\ \cline{2-4} 
																      & Read & Projeto(Subprojeto)                 & 1               \\ \cline{2-4} 
																      & Exit                & Projeto(Relatório)               & 1               \\ \hline
																      & Exit                & Projeto(Subprojeto)               & 1               \\ \hline
    \multicolumn{3}{|c|}{\textbf{TOTAL }}                                                                                                                                     & \textbf{5}               \\ \hline
    \end{tabular}
    \end{table}
    \pagebreak
    \subsection{Cadastrar Relatório Final}
  
      \begin{table}[!h]
      \centering
      \caption{Processo Funcional - Cadastrar Relatório Final}
      \label{pf_cadastrar_relatorio}
      \begin{tabular}{|c|c|c|c|}
      \hline
      \multicolumn{4}{|c|}{Cadastrar Relatório Final}                                                                                                                            \\ \hline
      \textbf{Evento disparador}                                                                                                        & \textbf{Movimentos} & \textbf{Grupo de dados} & \textbf{Pontos} \\ \hline
      \multirow{4}{*}{\begin{tabular}[c]{@{}c@{}}Usuário deseja registrar\\ um relatório final.\end{tabular}} & Entry               & Projeto   & 1               \\ \cline{2-4} 
																      & Read & Projeto                & 1               \\ \cline{2-4} 
																      & Read & Projeto(Relatório)                 & 1               \\ \cline{2-4} 
																      & Read & Projeto(Subprojeto)                 & 1               \\ \cline{2-4} 
																      & Write                & Projeto               & 1               \\ \cline{2-4}
																      & Write                & Projeto(Relatório)               & 1               \\ \cline{2-4}
																      & Write                & Projeto(Subprojeto)               & 1               \\ \cline{2-4}
																      & Exit                & Mensagens               & 1               \\ \hline
    \multicolumn{3}{|c|}{\textbf{TOTAL }}                                                                                                                                     & \textbf{8}               \\ \hline
    \end{tabular}
    \end{table}
    
        \subsection{Alterar Relatório Final}
  
      \begin{table}[!h]
      \centering
      \caption{Processo Funcional - Alterar Relatório Final}
      \label{pf_alterar_relatorio}
      \begin{tabular}{|c|c|c|c|}
      \hline
      \multicolumn{4}{|c|}{Alterar Relatório Final}                                                                                                                            \\ \hline
      \textbf{Evento disparador}                                                                                                        & \textbf{Movimentos} & \textbf{Grupo de dados} & \textbf{Pontos} \\ \hline
      \multirow{4}{*}{\begin{tabular}[c]{@{}c@{}}Usuário deseja alterar \\um relatório final enviado.\end{tabular}} & Entry               & Projeto   & 1               \\ \cline{2-4} 
																      & Read & Projeto                & 1               \\ \cline{2-4} 
																      & Read & Projeto(Relatório)                 & 1               \\ \cline{2-4} 
																      & Read & Projeto(Subprojeto)                 & 1               \\ \cline{2-4} 
																      & Write                & Projeto               & 1               \\ \cline{2-4}
																      & Write                & Projeto(Relatório)               & 1               \\ \cline{2-4}
																      & Write                & Projeto(Subprojeto)               & 1               \\ \cline{2-4}
																      & Exit                & Mensagens               & 1               \\ \hline
    \multicolumn{3}{|c|}{\textbf{TOTAL }}                                                                                                                                     & \textbf{8}               \\ \hline
    \end{tabular}
    \end{table}
  \pagebreak
          \subsection{Visualizar Relatório Final}
  
      \begin{table}[!h]
      \centering
      \caption{Processo Funcional - Visualizar Relatório Final}
      \label{pf_visualizar_relatorio}
      \begin{tabular}{|c|c|c|c|}
      \hline
      \multicolumn{4}{|c|}{Visualizar Relatório Final}                                                                                                                            \\ \hline
      \textbf{Evento disparador}                                                                                                        & \textbf{Movimentos} & \textbf{Grupo de dados} & \textbf{Pontos} \\ \hline
      \multirow{4}{*}{\begin{tabular}[c]{@{}c@{}}Usuário deseja visualizar\\ um relatório final enviado.\end{tabular}} & Entry               & Solicita a visualização   & 1               \\ \cline{2-4} 
																      & Read & Projeto                 & 1               \\ \cline{2-4} 
																      & Read & Projeto(Subprojeto)                 & 1               \\ \cline{2-4} 		
																      & Read & Projeto(Relatório) 	& 1               \\ \cline{2-4} 
																      & Exit & Projeto                 & 1               \\ \cline{2-4} 
																      & Exit & Projeto(Subprojeto)                 & 1               \\ \cline{2-4} 		
																      & Exit & Projeto(Relatório) 	& 1               \\ \cline{2-4} 
																      & Exit                & Mensagens               & 1               \\ \hline
    \multicolumn{3}{|c|}{\textbf{TOTAL }}                                                                                                                                     & \textbf{8}               \\ \hline
    \end{tabular}
    \end{table}
    
    
          \subsection{Visualizar Portaria de Habilitação}
  
      \begin{table}[!h]
      \centering
      \caption{Processo Funcional - Visualizar Portaria de Habilitação}
      \label{pf_visualizar_portaria}
      \begin{tabular}{|c|c|c|c|}
      \hline
      \multicolumn{4}{|c|}{Visualizar Portaria de Habilitação}                                                                                                                            \\ \hline
      \textbf{Evento disparador}                                                                                                        & \textbf{Movimentos} & \textbf{Grupo de dados} & \textbf{Pontos} \\ \hline
      \multirow{4}{*}{\begin{tabular}[c]{@{}c@{}}Usuário deseja visualizar a \\portaria de habilitação do projeto.\end{tabular}} & Entry               & Projeto   & 1               \\ \cline{2-4} 
																      & Read & Projeto(Portaria de Habilitação)     & 1               \\ \cline{2-4} 
																      & Exit                & Projeto               & 1               \\ \hline
    \multicolumn{3}{|c|}{\textbf{TOTAL }}                                                                                                                                     & \textbf{3}               \\ \hline
    \end{tabular}
    \end{table}
    
              \subsection{Visualizar o histórico de relatórios}
  
      \begin{table}[!h]
      \centering
      \caption{Processo Funcional - Visualizar o histórico de relatórios}
      \label{pf_visualizar_historico_relatorio}
      \begin{tabular}{|c|c|c|c|}
      \hline
      \multicolumn{4}{|c|}{Visualizar o histórico de relatórios}                                                                                                                            \\ \hline
      \textbf{Evento disparador}                                                                                                        & \textbf{Movimentos} & \textbf{Grupo de dados} & \textbf{Pontos} \\ \hline
      \multirow{4}{*}{\begin{tabular}[c]{@{}c@{}}Usuário deseja visualizar os relatórios.\end{tabular}} & Entry               & Projeto   & 1               \\ \cline{2-4} 
																      & Read & Projeto                 & 1               \\ \cline{2-4} 
																      & Read & Projeto(Relatório)                 & 1               \\ \cline{2-4} 
																      & Exit                & Projeto               & 1               \\ \hline
    \multicolumn{3}{|c|}{\textbf{TOTAL }}                                                                                                                                     & \textbf{4}               \\ \hline
    \end{tabular}
    \end{table}
    
    \vfill
    \pagebreak
  
              \subsection{Aceitar relatório}
  
      \begin{table}[!h]
      \centering
      \caption{Processo Funcional - Aceitar relatório}
      \label{pf_aceitar_relatorio}
      \begin{tabular}{|c|c|c|c|}
      \hline
      \multicolumn{4}{|c|}{Aceitar relatório}                                                                                                                            \\ \hline
      \textbf{Evento disparador}                                                                                                        & \textbf{Movimentos} & \textbf{Grupo de dados} & \textbf{Pontos} \\ \hline
      \multirow{4}{*}{\begin{tabular}[c]{@{}c@{}}Usuário deseja aceitar um relatório.\end{tabular}} & Entry               & Relatório   & 1               \\ \cline{2-4} 
																      & Write & Projeto(Relatório)                 & 1               \\ \cline{2-4} 
																      & Exit                & Mensagens & 1               \\ \hline
    \multicolumn{3}{|c|}{\textbf{TOTAL }}                                                                                                                                     & \textbf{3}               \\ \hline
    \end{tabular}
    \end{table}
                  \subsection{Recusar relatório}
  
      \begin{table}[!h]
      \centering
      \caption{Processo Funcional - Recusar relatório}
      \label{pf_recusar_relatorio}
      \begin{tabular}{|c|c|c|c|}
      \hline
      \multicolumn{4}{|c|}{Recusar relatório}                                                                                                                            \\ \hline
      \textbf{Evento disparador}                                                                                                        & \textbf{Movimentos} & \textbf{Grupo de dados} & \textbf{Pontos} \\ \hline
      \multirow{4}{*}{\begin{tabular}[c]{@{}c@{}}Usuário deseja aceitar um relatório.\end{tabular}} & Entry               & Relatório   & 1               \\ \cline{2-4} 
																      & Write & Projeto(Relatório)                 & 1               \\ \cline{2-4} 
																      & Exit                & Mensagens & 1               \\ \hline
    \multicolumn{3}{|c|}{\textbf{TOTAL }}                                                                                                                                     & \textbf{3}               \\ \hline
    \end{tabular}
    \end{table}
    
  \subsection{Visualizar Relatório de Acompanhamento por Projeto}
  
      \begin{table}[!h]
      \centering
      \caption{Processo Funcional - Visualizar Relatório de Acompanhamento por Projeto}
      \label{pf_visualizar_relatorio_projeto}
      \begin{tabular}{|c|c|c|c|}
      \hline
      \multicolumn{4}{|c|}{Visualizar Relatório de Acompanhamento por Projeto}                                                                                                                            \\ \hline
      \textbf{Evento disparador}                                                                                                        & \textbf{Movimentos} & \textbf{Grupo de dados} & \textbf{Pontos} \\ \hline
      \multirow{4}{*}{\begin{tabular}[c]{@{}c@{}}Usuário deseja visualizar o relatório\\  de acompanhamento por Projetos.\end{tabular}} & Entry               & Datas inicial e final   & 1               \\ \cline{2-4} 
																      & Read                & Projeto                 & 1               \\ \cline{2-4} 
																      & Exit                & Relatório do projeto    & 1               \\ \cline{2-4} 
																      & Exit                & Mensagens               & 1               \\ \hline
    \multicolumn{3}{|c|}{\textbf{TOTAL }}                                                                                                                                     & \textbf{4}               \\ \hline
    \end{tabular}
    \end{table}
    
  \vfill
  \pagebreak
  \subsection{Visualizar Relatório de Acompanhamento por Analista}
  
      \begin{table}[!h]
      \centering
      \caption{Processo Funcional - Visualizar Relatório de Acompanhamento por Analista}
      \label{pf_visualizar_relatorio_analista}
      \begin{tabular}{|c|c|c|c|}
      \hline
      \multicolumn{4}{|c|}{Visualizar Relatório de Acompanhamento por Analista}                                                                                                                           \\ \hline
      \textbf{Evento disparador}                                                                                                        & \textbf{Movimentos} & \textbf{Grupo de dados} & \textbf{Pontos} \\ \hline
      \multirow{4}{*}{\begin{tabular}[c]{@{}c@{}}Usuário deseja visualizar o relatório\\  de acompanhamento por Analista.\end{tabular}} & Entry               & Solicita o relatório    & 1               \\ \cline{2-4} 
																	& Read                & Projeto                 & 1               \\ \cline{2-4} 
																	& Exit                & Relatório do projeto    & 1               \\ \cline{2-4} 
																	& Exit                & Mensagens               & 1               \\ \hline
      \multicolumn{3}{|c|}{\textbf{TOTAL}}                                       
	& \textbf{4}      \\ \hline
      \end{tabular}
      \end{table}
      
   \subsection{Visualizar Projetos}
   
      \begin{table}[!h]
      \centering
      \caption{Processo funcional - Visualizar projetos}
      \label{pf_visualizar_projetos}
      \begin{tabular}{|c|c|c|c|}
      \hline
      \multicolumn{4}{|c|}{Visualizar Projetos}                                                                                                                                                                                                \\ \hline
      \textbf{Evento disparador}                                                                                    & \textbf{Movimentos} & \textbf{Grupo de dados}                                                          & \textbf{Pontos} \\ \hline
      \multirow{4}{*}{\begin{tabular}[c]{@{}c@{}}Usuário deseja gerenciar ou \\ analisar os projetos.\end{tabular}} & Entry               & \begin{tabular}[c]{@{}c@{}}Filtro para pesquisa\\ e solicita a ação\end{tabular} & 1               \\ \cline{2-4} 
														    & Read                & Projeto                                                                          & 1               \\ \cline{2-4} 
														    & Exit                & Projeto                                                                          & 1               \\ \cline{2-4} 
														    & Exit                & Mensagens                                                                        & 1               \\ \hline
      \multicolumn{3}{|c|}{\textbf{TOTAL}}                                                                                                                                                                          & \textbf{4}      \\ \hline
      \end{tabular}
      \end{table}
    
 
    \subsection{Detalhar projeto}
	
	\begin{table}[!h]
	\centering
	\caption{Processo funcional - Detalhar projeto}
	\label{pf_detalhar_projeto}
	\begin{tabular}{|c|c|c|c|}
	\hline
	\multicolumn{4}{|c|}{Detalhar Projeto}                                                                                                                                                        \\ \hline
	\textbf{Evento disparador}                                                                                                & \textbf{Movimentos} & \textbf{Grupo de dados}   & \textbf{Pontos} \\ \hline
	\multirow{3}{*}{\begin{tabular}[c]{@{}c@{}}Usuário deseja visualizar\\ mais informações sobre \\ o projeto.\end{tabular}} & Entry               & Solicita mais informações & 1               \\ \cline{2-4} 
																  & Read                & Projeto                   & 1               \\ \cline{2-4} 
																  & Exit                & Detalhes do Projeto       & 1               \\ \hline
	\multicolumn{3}{|c|}{\textbf{TOTAL}}                                                                                                                                        & \textbf{3}      \\ \hline
	\end{tabular}
	\end{table}
	
    \subsection{Validar versão do projeto}
    
	\begin{table}[!h]
	\centering
	\caption{Processo funcional - Validar versão do projeto}
	\label{pf_validar_versao}
	\begin{tabular}{|c|c|c|c|}
	\hline
	\multicolumn{4}{|c|}{Validar versão do Projeto}                                                                                                                             \\ \hline
	\textbf{Evento disparador}                                                                                & \textbf{Movimentos} & \textbf{Grupo de dados} & \textbf{Pontos} \\ \hline
	\multirow{2}{*}{\begin{tabular}[c]{@{}c@{}}Usuário deseja validar uma \\ versão do projeto.\end{tabular}} & Entry               & Solicita validação      & 1               \\ \cline{2-4} 
														  & Write               & Versão do Projeto       & 1               \\ \hline
	\multicolumn{3}{|c|}{\textbf{TOTAL}}                                                                                                                      & \textbf{2}      \\ \hline
	\end{tabular}
	\end{table}

    \subsection{Recusar versão do projeto}
    
	\begin{table}[!h]
	\centering
	\caption{Processo funcional - Recusar versão do projeto}
	\label{pf_recusar_versao}
	\begin{tabular}{|c|c|c|c|}
	\hline
	\multicolumn{4}{|c|}{Recusar versão do Projeto}                                                                                                                             \\ \hline
	\textbf{Evento disparador}                                                                                & \textbf{Movimentos} & \textbf{Grupo de dados} & \textbf{Pontos} \\ \hline
	\multirow{2}{*}{\begin{tabular}[c]{@{}c@{}}Usuário deseja recusar uma \\ versão do projeto.\end{tabular}} & Entry               & Solicita validação      & 1               \\ \cline{2-4} 
														  & Write               & Versão do Projeto       & 1               \\ \hline
	\multicolumn{3}{|c|}{\textbf{TOTAL}}                                                                                                                      & \textbf{2}      \\ \hline
	\end{tabular}
	\end{table}

    \subsection{Gerar Relatório – Base de dados geral}
    
	\begin{table}[!h]
	\centering
	\caption{Processo funcional - Gerar Relatório: Base de dados geral}
	\label{pf_relatorio_geral}
	\begin{tabular}{|c|c|c|c|}
	\hline
	\multicolumn{4}{|c|}{Gerar Relatório – Base de dados geral}                                                                                                                             \\ \hline
	\textbf{Evento disparador}                                                                                            & \textbf{Movimentos} & \textbf{Grupo de dados} & \textbf{Pontos} \\ \hline
	\multirow{6}{*}{\begin{tabular}[c]{@{}c@{}}Usuário deseja gerar um \\ relatório: Base de dados geral.\end{tabular}} & Entry                & Tipo de Relatório              & 1            \\ \cline{2-4} 
															      & Read               & Projeto                 	    & 1               \\ \cline{2-4}
															      & Read               & Projeto(Versão)                & 1               \\ \cline{2-4}
															      & Read               & Projeto(Subprojeto)            & 1               \\ \cline{2-4}
															      & Exit               & Relatório base de dados geral  & 1               \\ \cline{2-4}
															      & Exit               & Mensagens                      & 1               \\ \hline
	\multicolumn{3}{|c|}{\textbf{TOTAL}}                                                                                                                      & \textbf{6}      \\ \hline
	\end{tabular}
	\end{table}
	
    \pagebreak
    \subsection{Gerar Relatório – Base de acompanhamento}
    
	\begin{table}[!h]
	\centering
	\caption{Processo funcional - Gerar Relatório: Base de acompanhamento}
	\label{pf_relatorio_acompanhamento}
	\begin{tabular}{|c|c|c|c|}
	\hline
	\multicolumn{4}{|c|}{Gerar Relatório – Base de acompanhamento}                                                                                                                             \\ \hline
	\textbf{Evento disparador}                                                                                            & \textbf{Movimentos} & \textbf{Grupo de dados} & \textbf{Pontos}     \\ \hline
	\multirow{5}{*}{\begin{tabular}[c]{@{}c@{}}Usuário deseja gerar um \\ relatório: Base de acompanhamento.\end{tabular}} & Entry              & Tipo de Relatório		        & 1               \\ \cline{2-4} 
															      & Read               & Projeto                 	    	& 1               \\ \cline{2-4}
															      & Read               & Projeto(Atividade)                 & 1               \\ \cline{2-4}
															      & Exit               & Relatório base de acompanhamento   & 1               \\ \cline{2-4}
															      & Exit               & Mensagens                      	& 1               \\ \hline
	\multicolumn{3}{|c|}{\textbf{TOTAL}}                                                                                                                      & \textbf{5}      \\ \hline
	\end{tabular}
	\end{table}
	
	
     \subsection{Alterar responsável}
    
	\begin{table}[!h]
	\centering
	\caption{Processo funcional - Alterar responsável}
	\label{pf_alterar_responsavel}
	\begin{tabular}{|c|c|c|c|}
	\hline
	\multicolumn{4}{|c|}{Alterar responsável}                                                                                                                                                  \\ \hline
	\textbf{Evento disparador}                                                                                            & \textbf{Movimentos} & \textbf{Grupo de dados} & \textbf{Pontos}     \\ \hline
	\multirow{3}{*}{\begin{tabular}[c]{@{}c@{}}Usuário deseja alterar \\ responsável por projeto.\end{tabular}}           & Entry              & Analista do projeto       		& 1               \\ \cline{2-4}
															      & Write               & Projeto                 	    	& 1               \\ \cline{2-4}
															      & Exit               & Mensagens                      	& 1               \\ \hline
	\multicolumn{3}{|c|}{\textbf{TOTAL}}                                                                                                                      & \textbf{3}      \\ \hline
	\end{tabular}
	\end{table}
	
	
	
     \subsection{Suspender Análise}
    
	\begin{table}[!h]
	\centering
	\caption{Processo funcional - Suspender Análise}
	\label{pf_suspender_analise}
	\begin{tabular}{|c|c|c|c|}
	\hline
	\multicolumn{4}{|c|}{Suspender Análise}                                                                                                                                                  \\ \hline
	\textbf{Evento disparador}                                                                                            & \textbf{Movimentos} & \textbf{Grupo de dados} & \textbf{Pontos}     \\ \hline
	\multirow{3}{*}{\begin{tabular}[c]{@{}c@{}}Usuário deseja suspender \\ análise de um projeto.\end{tabular}}           & Entry              & Motivo da suspensão       		& 1               \\ \cline{2-4} 
															      & Write              & Projeto                 	    	& 1               \\ \cline{2-4}
															      & Exit               & Mensagens                      	& 1               \\ \hline
	\multicolumn{3}{|c|}{\textbf{TOTAL}}                                                                                                                      & \textbf{3}      \\ \hline
	\end{tabular}
	\end{table}
	
	
	
      \subsection{Solicitar informações para alterações de projeto}
    
	\begin{table}[!h]
	\centering
	\caption{Processo funcional - Solicitar informações para alterações de projeto}
	\label{pf_solicitar_alteração}
	\begin{tabular}{|c|c|c|c|}
	\hline
	\multicolumn{4}{|c|}{Solicitar informações para alterações de projeto}                                                                                                                                                  \\ \hline
	\textbf{Evento disparador}                                                                                            & \textbf{Movimentos} & \textbf{Grupo de dados} & \textbf{Pontos}     \\ \hline
	\multirow{5}{*}{\begin{tabular}[c]{@{}c@{}}Usuário deseja solicitar informações \\ para alteração de projeto.\end{tabular}}           & Entry              & solicitação de alteração      	& 1               \\ \cline{2-4} 
																	      & Write               & Conversa(Mensagem)                & 1               \\ \cline{2-4}
																	      & Write               & Projeto(Versão)               	& 1               \\ \cline{2-4}
																	      & Write               & Evento               	        & 1               \\ \cline{2-4}
																	      & Exit               & Mensagens                      	& 1               \\ \hline
	\multicolumn{3}{|c|}{\textbf{TOTAL}}                                                                                                                      & \textbf{5}      \\ \hline
	\end{tabular}
	\end{table}
	
      \vfill
      \pagebreak
      \subsection{Iniciar troca de mensagens}
    
	\begin{table}[!h]
	\centering
	\caption{Processo funcional -  Iniciar troca de mensagens}
	\label{pf_troca_mensagem}
	\begin{tabular}{|c|c|c|c|}
	\hline
	\multicolumn{4}{|c|}{ Iniciar troca de mensagens}                                                                                                                                                  \\ \hline
	\textbf{Evento disparador}                                                                                            & \textbf{Movimentos} & \textbf{Grupo de dados} & \textbf{Pontos}     \\ \hline
	\multirow{4}{*}{\begin{tabular}[c]{@{}c@{}}Usuário deseja iniciar uma \\ troca de mensagens.\end{tabular}}           & Entry              &\begin{tabular}[c]{@{}c@{}} solicitação de troca \\ de mensagens\end{tabular}  & 1               \\ \cline{2-4} 
																	      & Write               & Conversa(Mensagem)                & 1               \\ \cline{2-4}
																	      & Read              & Conversa(Mendagem)                 	& 1               \\ \cline{2-4}
																	      & Exit               & Mensagens                      	& 1               \\ \hline
	\multicolumn{3}{|c|}{\textbf{TOTAL}}                                                                                                                      & \textbf{4}      \\ \hline
	\end{tabular}
	\end{table}
	
	

      \subsection{Visualizar histórico de mensagens}
    
	\begin{table}[!h]
	\centering
	\caption{Processo funcional - Visualizar histórico de mensagens}
	\label{pf_historico_mensagem}
	\begin{tabular}{|c|c|c|c|}
	\hline
	\multicolumn{4}{|c|}{Visualizar histórico de mensagens}                                                                                                                                                  \\ \hline
	\textbf{Evento disparador}                                                                                            & \textbf{Movimentos} & \textbf{Grupo de dados} & \textbf{Pontos}     \\ \hline
	\multirow{3}{*}{\begin{tabular}[c]{@{}c@{}}Usuário deseja visualizar um \\ histórico de mensagens.\end{tabular}}           	      & Entry              &\begin{tabular}[c]{@{}c@{}} solicitação de visualização \\ de mensagens\end{tabular}  & 1               \\ \cline{2-4} 
																	      & Read              & Conversa(Mensagem)                	    	& 1               \\ \cline{2-4}
																	      & Exit               & Mensagens                      	& 1               \\ \hline
	\multicolumn{3}{|c|}{\textbf{TOTAL}}                                                                                                                      & \textbf{3}      \\ \hline
	\end{tabular}
	\end{table}
	
	
	
      \subsection{Encerrar troca de mensagens}
    
	\begin{table}[!h]
	\centering
	\caption{Processo funcional - Encerrar troca de mensagens}
	\label{pf_encerrar_mensagem}
	\begin{tabular}{|c|c|c|c|}
	\hline
	\multicolumn{4}{|c|}{Encerrar troca de mensagens}                                                                                                                                                  \\ \hline
	\textbf{Evento disparador}                                                                                            & \textbf{Movimentos} & \textbf{Grupo de dados} & \textbf{Pontos}     \\ \hline
	\multirow{3}{*}{\begin{tabular}[c]{@{}c@{}}Usuário deseja iniciar uma \\ troca de mensagens.\end{tabular}}           & Entry              &\begin{tabular}[c]{@{}c@{}} solicitação de troca \\ de mensagens\end{tabular}  & 1               \\ \cline{2-4} 
																	      & Write               & Conversa(Conversa)                 	    	& 1               \\ \cline{2-4}
																	      & Exit               & Mensagens                      	& 1               \\ \hline
	\multicolumn{3}{|c|}{\textbf{TOTAL}}                                                                                                                      & \textbf{3}      \\ \hline
	\end{tabular}
	\end{table}
	
	
	
	
	
	
	
\section{Suposições e observações}

    \begin{itemize}
      
      \item Para a contagem dos processos funcionais Visualizar "Relatório de Acompanhamento por Projeto",
	"Visualizar Relatório de Acompanhamento por Analista" e "Visualizar projetos" 
	(Tabelas \ref{pf_visualizar_relatorio_projeto}, \ref{pf_visualizar_relatorio_analista}, \ref{pf_visualizar_projetos}),
	a funcionalidade de gerar o relatório em PDF ou XLS não foi contada por apresentar uma saída com o mesmo grupo de 
	dados já utilizado na saída principal.
	
      \item O Processo funcional "Imprimir histórico de mensagens" não foi contado, pois apresenta as mesmas movimentações de dados do processo
      "Visualizar histórico de mensagens" (Tabela \ref{pf_historico_mensagem})
	
    \end{itemize}
