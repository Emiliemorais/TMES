\section{Data da contagem}

  A contagem por Pontos de Função COSMIC foi realizada no dia 27/09/2015.


\section{Nível de granularidade do RFU e nível de decomposição do software}
  
   Como a principal documentação utilizada como base para a contagem foram as especificações de casos de uso, 
   estes foram decompostos em requisitos funcionais menores que representassem todos os seus fluxos.

\vfill
\pagebreak
\section{Processos funcionais}
  
  Esta seção apresenta a contagem realizada para os processos funcionais identificados.
  
  \subsection{Visualizar Relatório de Acompanhamento por Projeto}
  
      \begin{table}[!h]
      \centering
      \caption{Processo Funcional - Visualizar Relatório de Acompanhamento por Projeto}
      \label{pf_visualizar_relatorio_projeto}
      \begin{tabular}{|c|c|c|c|}
      \hline
      \multicolumn{4}{|c|}{Visualizar Relatório de Acompanhamento por Projeto}                                                                                                                            \\ \hline
      \textbf{Evento disparador}                                                                                                        & \textbf{Movimentos} & \textbf{Grupo de dados} & \textbf{Pontos} \\ \hline
      \multirow{4}{*}{\begin{tabular}[c]{@{}c@{}}Usuário deseja visualizar o relatório\\  de acompanhamento por Projetos.\end{tabular}} & Entry               & Datas inicial e final   & 1               \\ \cline{2-4} 
																      & Read                & Projeto                 & 1               \\ \cline{2-4} 
																      & Exit                & Relatório do projeto    & 1               \\ \cline{2-4} 
																      & Exit                & Mensagens               & 1               \\ \hline
    \multicolumn{3}{|c|}{\textbf{TOTAL }}                                                                                                                                     & \textbf{4}               \\ \hline
    \end{tabular}
    \end{table}
  
  \subsection{Visualizar Relatório de Acompanhamento por Analista}
  
      \begin{table}[!h]
      \centering
      \caption{Processo Funcional - Visualizar Relatório de Acompanhamento por Analista}
      \label{pf_visualizar_relatorio_analista}
      \begin{tabular}{|c|c|c|c|}
      \hline
      \multicolumn{4}{|c|}{Visualizar Relatório de Acompanhamento por Analista}                                                                                                                           \\ \hline
      \textbf{Evento disparador}                                                                                                        & \textbf{Movimentos} & \textbf{Grupo de dados} & \textbf{Pontos} \\ \hline
      \multirow{4}{*}{\begin{tabular}[c]{@{}c@{}}Usuário deseja visualizar o relatório\\  de acompanhamento por Analista.\end{tabular}} & Entry               & Solicita o relatório    & 1               \\ \cline{2-4} 
																	& Read                & Projeto                 & 1               \\ \cline{2-4} 
																	& Exit                & Relatório do projeto    & 1               \\ \cline{2-4} 
																	& Exit                & Mensagens               & 1               \\ \hline
      \multicolumn{3}{|c|}{\textbf{TOTAL}}                                       
	& \textbf{4}      \\ \hline
      \end{tabular}
      \end{table}
      
   \subsection{Visualizar Projetos}
   
      \begin{table}[!h]
      \centering
      \caption{Processo funcional - Visualizar projetos}
      \label{pf_visualizar_projetos}
      \begin{tabular}{|c|c|c|c|}
      \hline
      \multicolumn{4}{|c|}{Visualizar Projetos}                                                                                                                                                                                                \\ \hline
      \textbf{Evento disparador}                                                                                    & \textbf{Movimentos} & \textbf{Grupo de dados}                                                          & \textbf{Pontos} \\ \hline
      \multirow{4}{*}{\begin{tabular}[c]{@{}c@{}}Usuário deseja gerenciar ou \\ analisar os projetos.\end{tabular}} & Entry               & \begin{tabular}[c]{@{}c@{}}Filtro para pesquisa\\ e solicita a ação\end{tabular} & 1               \\ \cline{2-4} 
														    & Read                & Projeto                                                                          & 1               \\ \cline{2-4} 
														    & Exit                & Projeto                                                                          & 1               \\ \cline{2-4} 
														    & Exit                & Mensagens                                                                        & 1               \\ \hline
      \multicolumn{3}{|c|}{\textbf{TOTAL}}                                                                                                                                                                          & \textbf{4}      \\ \hline
      \end{tabular}
      \end{table}
    
    \vfill
    \pagebreak
    \subsection{Detalhar projeto}
	
	\begin{table}[!h]
	\centering
	\caption{Processo funcional - Detalhar projeto}
	\label{pf_detalhar_projeto}
	\begin{tabular}{|c|c|c|c|}
	\hline
	\multicolumn{4}{|c|}{Detalhar Projeto}                                                                                                                                                        \\ \hline
	\textbf{Evento disparador}                                                                                                & \textbf{Movimentos} & \textbf{Grupo de dados}   & \textbf{Pontos} \\ \hline
	\multirow{3}{*}{\begin{tabular}[c]{@{}c@{}}Usuário deseja visualizar\\ mais informações sobre \\ o projeto.\end{tabular}} & Entry               & Solicita mais informações & 1               \\ \cline{2-4} 
																  & Read                & Projeto                   & 1               \\ \cline{2-4} 
																  & Exit                & Detalhes do Projeto       & 1               \\ \hline
	\multicolumn{3}{|c|}{\textbf{TOTAL}}                                                                                                                                        & \textbf{3}      \\ \hline
	\end{tabular}
	\end{table}
	
    \subsection{Validar versão do projeto}
    
	\begin{table}[!h]
	\centering
	\caption{Processo funcional - Validar versão do projeto}
	\label{pf_validar_versao}
	\begin{tabular}{|c|c|c|c|}
	\hline
	\multicolumn{4}{|c|}{Validar versão do Projeto}                                                                                                                             \\ \hline
	\textbf{Evento disparador}                                                                                & \textbf{Movimentos} & \textbf{Grupo de dados} & \textbf{Pontos} \\ \hline
	\multirow{2}{*}{\begin{tabular}[c]{@{}c@{}}Usuário deseja validar uma \\ versão do projeto.\end{tabular}} & Entry               & Solicita validação      & 1               \\ \cline{2-4} 
														  & Write               & Versão do Projeto       & 1               \\ \hline
	\multicolumn{3}{|c|}{\textbf{TOTAL}}                                                                                                                      & \textbf{2}      \\ \hline
	\end{tabular}
	\end{table}

    \subsection{Recusar versão do projeto}
    
	\begin{table}[!h]
	\centering
	\caption{Processo funcional - Recusar versão do projeto}
	\label{pf_recusar_versao}
	\begin{tabular}{|c|c|c|c|}
	\hline
	\multicolumn{4}{|c|}{Recusar versão do Projeto}                                                                                                                             \\ \hline
	\textbf{Evento disparador}                                                                                & \textbf{Movimentos} & \textbf{Grupo de dados} & \textbf{Pontos} \\ \hline
	\multirow{2}{*}{\begin{tabular}[c]{@{}c@{}}Usuário deseja recusar uma \\ versão do projeto.\end{tabular}} & Entry               & Solicita validação      & 1               \\ \cline{2-4} 
														  & Write               & Versão do Projeto       & 1               \\ \hline
	\multicolumn{3}{|c|}{\textbf{TOTAL}}                                                                                                                      & \textbf{2}      \\ \hline
	\end{tabular}
	\end{table}
	
\section{Suposições e observações}

    \begin{itemize}
      
      \item Para a contagem dos processos funcionais Visualizar "Relatório de Acompanhamento por Projeto",
	"Visualizar Relatório de Acompanhamento por Analista" e "Visualizar projetos" 
	(Tabelas \ref{pf_visualizar_relatorio_projeto}, \ref{pf_visualizar_relatorio_analista}, \ref{pf_visualizar_projetos}),
	a funcionalidade de gerar o relatório em PDF ou XLS não foi contada por apresentar uma saída com o mesmo grupo de 
	dados já utilizado na saída principal.
      
    \end{itemize}
