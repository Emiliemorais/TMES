\section{Data da contagem}

  A contagem por Pontos de Função foi realizada no dia 26/09/2015.

\section{Funções de Dados}

Esta seção apresenta os Arquivos Lógicos Internos (ALI) e Arquivos de Interface Externa (AIE) levantados, bem como a 
pontuação dos mesmos na contagem realizada.

  \subsection{ALI - Projeto}
  
      \begin{table}[!h]
      \centering
      \caption{ALI - Projeto}
      \label{ali_projeto}
      \begin{tabular}{|c|c|l|}
      \hline
      \multicolumn{3}{|c|}{\textbf{ALI - Projeto}}                                                      \\ \hline
      \multicolumn{1}{|l|}{RLR's} & Tipo                         & \multicolumn{1}{c|}{DER's}  \\ \hline
      \multirow{8}{*}{1}          & \multirow{8}{*}{Obrigatório} & ID do projeto               \\ \cline{3-3} 
				  &                              & Nome do projeto             \\ \cline{3-3} 
				  &                              & Titular do projeto (PJ)     \\ \cline{3-3} 
				  &                              & Data de submissão           \\ \cline{3-3} 
				  &                              & Data de protocolo           \\ \cline{3-3} 
				  &                              & Situação                    \\ \cline{3-3} 
				  &                              & É consórcio                 \\ \cline{3-3} 
				  &                              & Responsável pelo projeto    \\ \hline
      \multirow{4}{*}{2}        & \multirow{4}{*}{Obrigatório} & Número da versão do projeto \\ \cline{3-3} 
				  &                              & Data e hora da versão       \\ \cline{3-3} 
				  &                              & Tipo de validação da versão \\ \cline{3-3} 
				  &                              & Responsável pela versão     \\ \hline
      \multirow{11}{*}{3}         &\multirow{11}{*}{Opcional}    & ID do subprojeto               \\ \cline{3-3} 
				  &                              & Título do subprojeto             \\ \cline{3-3} 
				  &                              & Valor do subprojeto      \\ \cline{3-3} 
				  &                              & Tipo do subprojeto          \\ \cline{3-3} 
				  &                              & Descrição final do subprojeto          \\ \cline{3-3} 
				  &                              & Data de início do subprojeto                  \\ \cline{3-3} 
				  &                              & Data de conclusão do subprojeto                 \\ \cline{3-3}
				  &				 & Planilha de execução físico-financeira \\ \cline{3-3} 
				  &  				 & Informações georreferenciadas \\ \cline{3-3}         
				  &                              & Benefícios alcançados    \\ \hline
      \end{tabular}
      \end{table}

  \subsection{AIE - Usuário interno}
      
      Os usuários internos são funcionários do Ministério das Comunicações, onde foi considerado que os dados deles
      eram obtidos com o acesso a um sistema de funcionários do próprio ministério, o que justifica a definição como AIE.
      Como não foram encontrados na documentação os dados que são armazenados neste sistema externo,
      foram considerados apenas os atributos mais relevantes.
      
      \begin{table}[!h]
      \centering
      \caption{AIE - Usuário interno}
      \label{aie_usuario_interno}
      \begin{tabular}{|c|c|l|}
      \hline
      \multicolumn{3}{|c|}{AIE - Usuário interno}                                             \\ \hline
      \multicolumn{1}{|l|}{RLR's} & Tipo                         & \multicolumn{1}{c|}{DER's} \\ \hline
      \multirow{5}{*}{1}          & \multirow{5}{*}{Obrigatório} & Nome                       \\ \cline{3-3} 
				  &                              & CPF                        \\ \cline{3-3} 
				  &                              & Login                      \\ \cline{3-3} 
				  &                              & Senha                      \\ \cline{3-3}
                                  &		                 & Cargo                      \\ \hline
      \end{tabular}
      \end{table}
      
  
  \subsection{ALI - Usuário externo}
    
    Os usuários externos são caracterizados como proponentes de projetos no sistema.
    Os dados do ALI - Usuário externo foram retirados do caso de uso UC01 - Login do Usuário, que está fora do escopo 
    de contagem, mas foi considerado para o levantamento dos DERs.
    
    \begin{table}[!h]
    \centering
    \caption{ALI - Usuário externo}
    \label{ali_usuario_externo}
    \begin{tabular}{|c|c|l|}
    \hline
    \multicolumn{3}{|c|}{ALI - Usuário Externo}                                              \\ \hline
    \multicolumn{1}{|l|}{RLR's} & Tipo                          & \multicolumn{1}{c|}{DER's} \\ \hline
    \multirow{10}{*}{1}         & \multirow{10}{*}{Obrigatório} & Nome                       \\ \cline{3-3} 
				&                               & CPF                        \\ \cline{3-3} 
				&                               & Login                      \\ \cline{3-3} 
				&                               & Senha                      \\ \cline{3-3} 
				&                               & RG                         \\ \cline{3-3} 
				&                               & Telefone                   \\ \cline{3-3} 
				&                               & Celular                    \\ \cline{3-3} 
				&                               & Data de nascimento         \\ \cline{3-3} 
				&                               & E-mail                     \\ \cline{3-3} 
				&                               & Sexo                       \\ \hline
    \end{tabular}
    \end{table}
  
  \pagebreak
  \subsection{ALI - Conversa}
    
      Uma conversa representa uma instância de comunicação entre um proponente e o Ministério das Comunicações,
      onde pode haver várias mensagens.
    
      \begin{table}[!h]
      \centering
      \caption{ALI - Conversa}
      \label{my-label}
      \begin{tabular}{|c|c|l|}
      \hline
      \multicolumn{3}{|c|}{ALI - Conversa}                                                    \\ \hline
      \multicolumn{1}{|l|}{RLR's} & Tipo                         & \multicolumn{1}{c|}{DER's} \\ \hline
      \multirow{3}{*}{1}          & \multirow{3}{*}{Obrigatório - Mensagem} & Conteúdo da mensagem       \\ \cline{3-3} 
				  &                              & Data e hora                \\ \cline{3-3} 
				  &                              & Remetente                  \\ \hline
      \multirow{2}{*}{2}          & \multirow{2}{*}{Obrigatório - Conversa} & Status da conversa         \\ \cline{3-3} 
				  &                              & Projeto relacionado        \\ \hline
      \end{tabular}
      \end{table}
      
  \subsection{Pontuação das funções de dados}
  
      \begin{table}[!h]
      \centering
      \caption{Informações sobre as funções de dados}
      \label{funcoes_dados}
	\begin{tabular}{|p{0.12\linewidth}|p{0.20\linewidth}|p{0.14\linewidth}|}
	\hline
	\textbf{Função} & \textbf{Complexidade} & \textbf{Pontos de função (PF)} \\
	\hline
	Projeto &  &  \\
	\hline
	Usuário interno &   & \\
	\hline
	Usuário externo &   & \\
	\hline
	Conversa &  & \\
	\hline
	\end{tabular}
      \end{table}

\vfill
\pagebreak
\section{Funções de Transação}
  
 Esta seção apresenta as funções de transação identificadas, bem como suas classificações e pontuação na contagem realizada.
  
  \subsection{Processos de Entradas Externas}
  
    \subsubsection{EE - Suspender análise}

\begin{table}[!h]
\centering
\caption{EE - Suspender análise}
\label{ee_suspender_analise}
\begin{tabular}{|l|l|l|}
\hline
\multicolumn{3}{|c|}{EE - Suspender análise}                          \\ \hline
ALR                           & DER's                         & Total \\ \hline
\multirow{2}{*}{Projeto}      & Motivo da suspensão           & 1     \\ \cline{2-3} 
                              & Situação                      & 1     \\ \hline
\multirow{5}{*}{Evento}       & Tipo da atividade             & 1     \\ \cline{2-3} 
                              & Data                          & 1     \\ \cline{2-3} 
                              & Hora                          & 1     \\ \cline{2-3} 
                              & Nome do usuário que executou  & 1     \\ \cline{2-3} 
                              & Projeto que sofreu a ação     & 1     \\ \hline
\multirow{2}{*}{DER's extras} & Capacidade de iniciar ação    & 1     \\ \cline{2-3} 
                              & Capacidade de emitir mensagem & 1     \\ \hline
\multicolumn{2}{|c|}{\textbf{TOTAL DE DER's}}                          & \textbf{9}     \\ \hline
\end{tabular}
\end{table}

 \subsubsection{EE - Iniciar troca de mensagens}
\begin{table}[!h]
\centering
\caption{EE - Iniciar troca de mensagens}
\label{ee_iniciar_troca}
\begin{tabular}{|l|l|l|}
\hline
\multicolumn{3}{|c|}{EE - Iniciar troca de mensagens}                 \\ \hline
ALR                           & DER's                         & Total \\ \hline
\multirow{4}{*}{Conversa}     & Conteúdo                      & 1     \\ \cline{2-3} 
                              & Data e hora                   & 1     \\ \cline{2-3} 
                              & Remetente                     & 1     \\ \cline{2-3} 
                              & Status                        & 1     \\ \hline
\multirow{4}{*}{Evento}       & Tipo da atividade             & 1     \\ \cline{2-3} 
                              & Data e hora                   & 1     \\ \cline{2-3} 
                              & Nome do usuário que executou  & 1     \\ \cline{2-3} 
                              & Projeto que sofreu a ação     & 1     \\ \hline
\multirow{2}{*}{DER's extras} & Capacidade de iniciar ação    & 1     \\ \cline{2-3} 
                              & Capacidade de emitir mensagem & 1     \\ \hline
\multicolumn{2}{|c|}{TOTAL DE DER's}                          & 10    \\ \hline
\end{tabular}
\end{table}
  
 
\vfill
\pagebreak
  \subsubsection{EE - Enviar mensagens}
\begin{table}[!h]
\centering
\caption{EE - Enviar mensagens}
\label{ee_enviar_mensagens}
\begin{tabular}{|l|l|l|}
\hline
\multicolumn{3}{|c|}{EE - Enviar mensagens}                 \\ \hline
ALR                           & DER's                         & Total \\ \hline
\multirow{3}{*}{Conversa}     & Conteúdo                      & 1     \\ \cline{2-3} 
                              & Data e hora                   & 1     \\ \cline{2-3} 
                              & Remetente                     & 1     \\ \cline{2-3} \hline
\multirow{2}{*}{DER's extras} & Capacidade de iniciar ação    & 1     \\ \cline{2-3} 
                              & Capacidade de emitir mensagem & 1     \\ \hline
\multicolumn{2}{|c|}{TOTAL DE DER's}                          & 5    \\ \hline
\end{tabular}
\end{table}

   \subsubsection{EE - Encerrar troca de mensagens}
\begin{table}[!h]
\centering
\caption{EE - Encerrar troca de mensagens}
\label{ee_enviar_mensagens}
\begin{tabular}{|l|l|l|}
\hline
\multicolumn{3}{|c|}{EE - Encerrar troca de mensagens}                 \\ \hline
ALR                           & DER's                         & Total \\ \hline
Conversa     & Status & 1     \\ \hline 
\multirow{2}{*}{DER's extras} & Capacidade de iniciar ação    & 1     \\ \cline{2-3} 
                              & Capacidade de emitir mensagem & 1     \\ \hline
\multicolumn{2}{|c|}{TOTAL DE DER's}                          & 3    \\ \hline
\end{tabular}
\end{table}

  \subsubsection{EE - Upload Portaria de Habilitação}
\begin{table}[!h]
\centering
\caption{EE - Upload Portaria de Habilitação}
\label{ee_upload_portaria_de_habilitacao}
\begin{tabular}{|c|l|c|}
\hline
\multicolumn{3}{|c|}{EE - Upload Portaria de Habilitação}                    \\ \hline
ALR                      & \multicolumn{1}{c|}{DER's} & Total \\ \hline
\multirow{6}{*}{Projeto} & Data da portaria                         & 1     \\ \cline{2-3} 
                         & Data de publicação            & 1     \\ \cline{2-3} 
                         & Nº da portaria      & 1     \\ \cline{2-3} 
                         & Portaria de habilitação              & 1     \\ \cline{2-3}
			 & Data de Submissão        & 1     \\ \cline{2-3} 
                         & Situação & 1     \\ \hline
\multirow{2}{*}{DER's extras}             & Capacidade de iniciar ação & 1      \\ \cline{2-3} 
			 & Capacidade de emitir mensagem & 1\\ \hline
\multicolumn{2}{|c|}{\textbf{TOTAL DE DER's}}                  & \textbf{8}     \\ \hline
\end{tabular}
\end{table}



\vfill
\pagebreak
  \subsubsection{EE - Submeter Relatório Parcial}

\begin{table}[!h]
\centering
\caption{EE - Submeter Relatório Parcial}
\label{ee_submeter_relatorio_parcial}
\begin{tabular}{|l|l|l|}
\hline
\multicolumn{3}{|c|}{EE - Submeter Relatório Parcial}                                                                 \\ \hline
ALR                           & DER's                                                & Total              \\ \hline
\multirow{6}{*}{Projeto}    & Desenho esquemático  & 1                  \\ \cline{2-3} 
                             & Situação do relatório & 1 \\ \cline{2-3}
			      & Subprojeto & 1 \\ \cline{2-3}
                              & Planilha de execução físico-financeira               & 1                  \\ \cline{2-3} 
                              & Informações georreferenciadas                        & 1                  \\ \cline{2-3} 
                              & Outros  & 1                  \\ \hline
\multirow{6}{*}{DER's extras} & Houve execução física e/ou financeira do projeto? & 1                  \\ \cline{2-3} 
                              & Projeto como executado & 1 \\ \cline{2-3}
                              & Houve execução física e/ou financeira do subprojeto?    & 1                  \\ \cline{2-3} 
			      & Não alterado                                         & 1                  \\ \cline{2-3} 
                              & Capacidade de iniciar ação                           & 1\\ \cline{2-3}
                              & Capacidade de emitir mensagem                        &1                    \\ \hline
\multicolumn{2}{|c|}{\textbf{TOTAL DE DER's}}                                                 & \textbf{12}                 \\ \hline
\end{tabular}
\end{table}

  \subsubsection{EE - Cadastrar Relatório Final}
  
\begin{table}[!h]
\centering
\caption{EE - Cadastrar Relatório Final}
\label{ee_cadastrar_relatório_final}
\begin{tabular}{|l|l|l|}
\hline
\multicolumn{3}{|c|}{EE - Cadastrar Relatório Final}                                                                 \\ \hline
ALR                           & DER's                                                & Total              \\ \hline
\multirow{13}{*}{Projeto}     & Descrição final do projeto             & 1 \\ \cline{2-3}
                             & Data de conclusão do projeto           & 1 \\ \cline{2-3}
                             & Desenho esquemático                    & 1 \\ \cline{2-3} 
			      & Subprojeto                             & 1 \\ \cline{2-3}
                             & Descrição final do subprojeto          & 1 \\ \cline{2-3}
                             & Benefícios alcançados                  & 1 \\ \cline{2-3}
                             & Data de conclusão do subprojeto        & 1 \\ \cline{2-3}
                             & Planilha de execução físico-financeira & 1 \\ \cline{2-3}
                             & Informações georreferenciadas          & 1 \\ \cline{2-3}
                             & Notas fiscais                          & 1 \\ \cline{2-3}
                             & Outros                                 & 1 \\ \cline{2-3}
                             & Item                                   & 1 \\ \cline{2-3}
                             & Nome                                   & 1 \\ \cline{2-3} \hline
\multirow{2}{*}{DER's extras} & Capacidade de iniciar ação             & 1 \\ \cline{2-3}
                 & Capacidade de emitir mensagem& 1 \\ \hline

\multicolumn{2}{|c|}{\textbf{TOTAL DE DER's}}                                                 & \textbf{15}                 \\ \hline
\end{tabular}
\end{table}


\vfill
\pagebreak

  \subsubsection{EE - Alterar Relatório Final}
  \begin{table}[!h]
\centering
\caption{EE - Alterar Relatório Final}
\label{ee_alterar_relatorio_final}
\begin{tabular}{|l|l|l|}
\hline
\multicolumn{3}{|c|}{EE - Alterar Relatório Final}                                                                 \\ \hline
ALR                           & DER's                                                & Total              \\ \hline
\multirow{13}{*}{Projeto}     & Descrição final do projeto             & 1 \\ \cline{2-3}
                             & Data de conclusão do projeto           & 1 \\ \cline{2-3}
                             & Desenho esquemático                    & 1 \\ \cline{2-3} 
			      & Subprojeto                             & 1 \\ \cline{2-3}
                             & Descrição final do subprojeto          & 1 \\ \cline{2-3}
                             & Benefícios alcançados                  & 1 \\ \cline{2-3}
                             & Data de conclusão do subprojeto        & 1 \\ \cline{2-3}
                             & Planilha de execução físico-financeira & 1 \\ \cline{2-3}
                             & Informações georreferenciadas          & 1 \\ \cline{2-3}
                             & Notas fiscais                          & 1 \\ \cline{2-3}
                             & Outros                                 & 1 \\ \cline{2-3}
                             & Item                                   & 1 \\ \cline{2-3}
                             & Nome                                   & 1 \\ \cline{2-3} \hline
\multirow{2}{*}{DER's extras} & Capacidade de iniciar ação             & 1 \\ \cline{2-3}
                 & Capacidade de emitir mensagem& 1 \\ \hline
\multicolumn{2}{|c|}{\textbf{TOTAL DE DER's}}                                                 & \textbf{15}                 \\ \hline
\end{tabular}
\end{table}


  \subsubsection{EE - Recusar Relatório }
    \begin{table}[!h]
\centering
\caption{EE - Recusar Relatorio}
\label{ee_recusar_relatorio_parcial}
\begin{tabular}{|l|l|l|}
\hline
\multicolumn{3}{|c|}{EE - Recusar Relatorio}           \\ \hline
ALR                      & DER's                      & Total         \\ \hline
Projeto                  & Situação do relatório      & 1             \\\hline
  \multirow{2}{*}{DER's extras} & Capacidade de iniciar ação             & 1 \\ \cline{2-3}
                 & Capacidade de emitir mensagem& 1 \\ \hline
\multicolumn{2}{|c|}{\textbf{TOTAL DE DER's}}                & \textbf{3} \\ \hline
\end{tabular}
\end{table}

  \subsubsection{EE - Validar versão do projeto}
   
      \begin{table}[!h]
      \centering
      \caption{EE - Validar versão do projeto}
      \label{ee_validar_versao_projeto}
      \begin{tabular}{|c|l|c|}
      \hline
      \multicolumn{3}{|c|}{EE - Validar versão do projeto}                        \\ \hline
      ALR                      & \multicolumn{1}{c|}{DER's}  & Total              \\ \hline
      \multirow{4}{*}{Projeto} & Número da Versão            & \multirow{4}{*}{4} \\ \cline{2-2}
			      & Data e hora                 &                    \\ \cline{2-2}
			      & Tipo de validação da versão &                    \\ \cline{2-2}
			      & Situação                    &                    \\ \hline
      DERs extras              & Capacidade de iniciar ação  & 1                  \\ \hline
      \multicolumn{2}{|c|}{\textbf{TOTAL DE DER's}}          & \textbf{5}         \\ \hline
      \end{tabular}
      \end{table}
   
   
\vfill
\pagebreak
   \subsubsection{EE - Recusar versão do projeto}
   
      \begin{table}[!h]
      \centering
      \caption{EE - Recusar versão do projeto}
      \label{ee_resusar_versao_projeto}
      \begin{tabular}{|c|l|c|}
      \hline
      \multicolumn{3}{|c|}{EE - Recusar versão do projeto}                        \\ \hline
      ALR                      & \multicolumn{1}{c|}{DER's}  & Total              \\ \hline
      \multirow{4}{*}{Projeto} & Número da Versão            & \multirow{4}{*}{4} \\ \cline{2-2}
			      & Data e hora                 &                    \\ \cline{2-2}
			      & Tipo de validação da versão &                    \\ \cline{2-2}
			      & Situação                    &                    \\ \hline
      DERs extras              & Capacidade de iniciar ação  & 1                  \\ \hline
      \multicolumn{2}{|c|}{\textbf{TOTAL DE DER's}}          & \textbf{5}         \\ \hline
      \end{tabular}
      \end{table}
      
   \subsubsection{EE - Alterar responsável}
      
      \begin{table}[!h]
      \centering
      \caption{EE - Alterar responsável}
      \label{ee_alterar_responsavel}
      \begin{tabular}{|c|l|c|}
      \hline
      \multicolumn{3}{|c|}{EE - Alterar responsável}                              \\ \hline
      ALR                          & \multicolumn{1}{c|}{DER's}    & Total        \\ \hline
      Projeto                      & Nome do responsável           & 1            \\ \hline
      \multirow{2}{*}{DERs extras} & Capacidade de iniciar ação    & 1            \\ \cline{2-3} 
				  & Capacidade de emitir mensagem & 1            \\ \hline
       \multicolumn{2}{|c|}{\textbf{TOTAL DE DER's}}          & \textbf{3}         \\ \hline
      \end{tabular}
      \end{table}
      
        \subsubsection{EE - Aceitar Relatorio}
  \begin{table}[!h]
\centering
\caption{EE - Aceitar Relatorio}
\label{ee_aceitar_relatorio_parcial}
\begin{tabular}{|l|l|l|}
\multicolumn{3}{c}{EE - Aceitar Relatorio}           \\\hline
ALR                      & DER's                      & Total         \\ \hline
Projeto                  & Situação do relatório      & 1             \\\hline
  \multirow{2}{*}{DER's extras} & Capacidade de iniciar ação             & 1 \\ \cline{2-3}
                 & Capacidade de emitir mensagem& 1 \\ \hline
\multicolumn{2}{|c|}{\textbf{TOTAL DE DER's}}                    & \textbf{3} \\ \hline
\end{tabular}
\end{table}

  
  \subsection{Processos de Saídas Externas}
    
        \subsubsection{SE - Visualizar Relatório de Acompanhamento das Atividades do Projeto}
  
    Este processo elementar foi classificado como Saída Externa porque, segundo a Regra de Negócio RN63, neste relatório
    deve conter a quantidade de projetos por situação, que é calculada na hora de gerar o relatório.
    
      \begin{table}[!h]
      \centering
      \caption{SE - Visualizar Relatório de Acompanhamento das Atividades do Projeto}
      \label{se_visualizar_relatorio_acompanhamento}
      \begin{tabular}{|c|l|c|}
      \hline
      \multicolumn{3}{|c|}{SE - Visualizar Relatório de Acompanhamento das Atividades do Projeto} \\ \hline
      ALR                              & \multicolumn{1}{c|}{DER's}       & Total                 \\ \hline
      \multirow{2}{*}{Projeto}         & Situação                         & 1                     \\ \cline{2-3} 
				      & Data de submissão                & 1                     \\ \hline
      Usuário                          & Nome do analista responsável     & 1                     \\ \hline
      \multirow{4}{*}{DERs extras}     & Capacidade de iniciar ação       & \multirow{4}{*}{4}    \\ \cline{2-2}
				      & Mensagem de erro                 &                       \\ \cline{2-2}
				      & Data inicial                     &                       \\ \cline{2-2}
				      & Data final                       &                       \\ \hline
      \multicolumn{2}{|c|}{\textbf{TOTAL DE DER's}}                       & \textbf{7}            \\ \hline
      \end{tabular}
      \end{table}
    
  \subsubsection{SE - Visualizar Relatório de Acompanhamento das Atividades por Analista}
      
      Este processo elementar foi classificado como Saída Externa porque, segundo a Regra de Negócio RN63, neste relatório
    deve conter a quantidade de projetos por situação, que é calculada na hora de gerar o relatório.
      
      \begin{table}[!h]
      \centering
      \caption{SE - Visualizar Relatório de Acompanhamento das Atividades por Analista}
      \label{se_visualizar_relatorio_acompanhamento_analista}
      \begin{tabular}{|c|l|c|}
      \hline
      \multicolumn{3}{|c|}{SE - Visualizar Relatório de Acompanhamento das Atividades por Analista} \\ \hline
      ALR                               & \multicolumn{1}{c|}{DER's}       & Total                  \\ \hline
      Projeto                           & Situação                         & 1                      \\ \hline
      Usuário                           & Nome do analista responsável     & 1                      \\ \hline
      \multirow{3}{*}{DERs extras}      & Capacidade de iniciar ação       & \multirow{3}{*}{3}     \\ \cline{2-2}
					& Mensagem de erro                 &                        \\ \cline{2-2}
					& Total de projetos                &                        \\ \hline
      \textbf{TOTAL DE DER's}           &                                  & \textbf{5}             \\ \hline
      \end{tabular}
      \end{table}
  

    
  \subsubsection{SE - Gerar Relatório: Base de Dados Geral}
   
    \begin{table}[!h]
    \centering
    \caption{SE - Gerar Relatório: Base de Dados Geral}
    \label{se_base_geral}
    \begin{tabular}{|c|l|c|}
    \hline
    \multicolumn{3}{|c|}{SE - Gerar Relatório: Base de Dados Geral}                 \\ \hline
    ALR                         & \multicolumn{1}{c|}{DER's}    & Total                \\ \hline
    \multirow{21}{*}{Projeto}   & Projeto Original              & \multirow{21}{*}{22} \\ \cline{2-2}
				& Título do Projeto             &                      \\ \cline{2-2}
				& Número Protocolo              &                      \\ \cline{2-2}
				& ID do Projeto                 &                      \\ \cline{2-2}
				& Número da Versão              &                      \\ \cline{2-2}
				& PJ Titular do Projeto         &                      \\ \cline{2-2}
				& CNPJ                          &                      \\ \cline{2-2}
				& Ano de Constituição           &                      \\ \cline{2-2}
				& Capital Social                &                      \\ \cline{2-2}
				& Representante Legal           &                      \\ \cline{2-2}
				& Data de Submissão             &                      \\ \cline{2-2}
				& Data de Protocolo             &                      \\ \cline{2-2}
				& Situação                      &                      \\ \cline{2-2}
				& Situação da Versão            &                      \\ \cline{2-2}
				& Situação interna              &                      \\ \cline{2-2}
				& É consórcio empresarial       &                      \\ \cline{2-2}
				& Data de Início                &                      \\ \cline{2-2}
				& Data de Conclusão             &                      \\ \cline{2-2}
				& Valor do projeto              &                      \\ \cline{2-2}
				& Munucípio                     &                      \\ \cline{2-2}
				& UF                            &                      \\ \hline
    Usuário                     & Analista                      & 1                    \\ \hline
    \multirow{4}{*}{Subprojeto} & ID Subprojeto                 & \multirow{4}{*}{4}   \\ \cline{2-2}
				& Título do Subprojeto          &                      \\ \cline{2-2}
				& Tipo de Subprojeto            &                      \\ \cline{2-2}
				& Valor do Subprojeto           &                      \\ \hline
    DERs extras                 & Capacidade de iniciar ação    & 1                    \\ \hline
    \multicolumn{1}{|l|}{}      & Capacidade de emitir mensagem & 1                    \\ \hline
    \multicolumn{2}{|c|}{\textbf{TOTAL DE DER's}}          & \textbf{29}         \\ \hline
    \end{tabular}
    \end{table}
  
\vfill
\pagebreak
  \subsubsection{SE - Gerar Relatório: Base de Acompanhamento}
   
    \begin{table}[!h]
    \centering
    \caption{SE - Gerar Relatório: Base de Acompanhamento}
    \label{se_base_acompanhamento}
    \begin{tabular}{|c|l|c|}
    \hline
    \multicolumn{3}{|c|}{SE - Gerar Relatório: Base de Acompanhamento}            \\ \hline
    ALR                      & \multicolumn{1}{c|}{DER's}    & Total              \\ \hline
    \multirow{5}{*}{Projeto} & Projeto Original              & \multirow{5}{*}{5} \\ \cline{2-2}
			    & Título do Projeto             &                    \\ \cline{2-2}
			    & Número Protocolo              &                    \\ \cline{2-2}
			    & Data Atividade                &                    \\ \cline{2-2}
			    & Atividade                     &                    \\ \hline
    Usuário                  & Analista                      & 1                  \\ \hline
    DERs extras              & Capacidade de iniciar ação    & 1                  \\ \hline
    \multicolumn{1}{|l|}{}   & Capacidade de emitir mensagem & 1                  \\ \hline
    \multicolumn{2}{|c|}{\textbf{TOTAL DE DER's}}          & \textbf{8}         \\ \hline
    \end{tabular}
    \end{table}

  \subsection{Processos de Consultas Externas}
    
    \subsubsection{CE - Visualizar histórico de mensagens}
\begin{table}[!h]
\centering
\caption{CE - Visualizar histórico de mensagens}
\label{ee_enviar_mensagens}
\begin{tabular}{|l|l|l|}
\hline
\multicolumn{3}{|c|}{CE - Visualizar histórico de mensagens}                 \\ \hline
ALR                           & DER's                         & Total \\ \hline
\multirow{3}{*}{Conversa}     & Conteúdo                      & 1     \\ \cline{2-3} 
                              & Data e hora                   & 1     \\ \cline{2-3} 
                              & Remetente                     & 1     \\ \cline{2-3} \hline
DER's extras & Capacidade de iniciar ação    & 1     \\ \hline
\multicolumn{2}{|c|}{TOTAL DE DER's}                          & \textbf{4}    \\ \hline
\end{tabular}
\end{table}
 
  \subsubsection{CE - Listar projetos}

\begin{table}[!h]
\centering
\caption{CE - Listar projetos}
\label{ce_listar_projeto}
\begin{tabular}{|c|l|c|}
\hline
\multicolumn{3}{|c|}{CE - Listar projetos}                    \\ \hline
ALR                      & \multicolumn{1}{c|}{DER's} & Total \\ \hline
\multirow{5}{*}{Projeto} & ID                         & 1     \\ \cline{2-3} 
                         & Nome do projeto            & 1     \\ \cline{2-3} 
                         & PJ titular do projeto      & 1     \\ \cline{2-3} 
                         & Versão válida              & 1     \\ \cline{2-3} 
                         & Situação do projeto        & 1     \\ \hline
DER Extra                & Capacidade de iniciar ação & 1     \\ \hline
\multicolumn{2}{|c|}{\textbf{TOTAL DE DER's}}                  & \textbf{6}     \\ \hline
\end{tabular}
\end{table}


  \subsubsection{CE - Visualizar Relatório Parcial}
  
  \begin{table}[!h]
\centering
\caption{CE - Visualizar Relatório Parcial}
\label{ce_visualizar_relatorio_parcial}
\begin{tabular}{|l|l|l|}
\hline
\multicolumn{3}{|c|}{CE - Visualizar Relatório Parcial}                                                                 \\ \hline
ALR                           & DER's                                                & Total              \\ \hline
\multirow{7}{*}{Projeto}     & Desenho esquemático                                  & 1                  \\ \cline{2-3}
			      & Subprojeto                                           & 1                  \\ \cline{2-3} 
                              & Planilha de execução físico-financeira               & 1                  \\ \cline{2-3} 
                              & Informações georreferenciadas                        & 1                  \\ \cline{2-3} 
                              & Outros                                               & 1                  \\ \cline{2-3} 
                              & Projeto como executado                               & 1                  \\ \hline
DER's extras		      & Capacidade de iniciar ação                           & 1 \\ \hline
\multicolumn{2}{|c|}{\textbf{TOTAL DE DER's}}                                                 & \textbf{7}                 \\ \hline
\end{tabular}
\end{table}


\vfill
\pagebreak


  \subsubsection{CE - Visualizar Relatório Final}
 \begin{table}[!h]
 \centering
  \caption{CE - Visualizar Relatório Final}
\label{ee_visualizar_relatorio_final}
\begin{tabular}{|l|l|l|}
\hline
\multicolumn{3}{|c|}{CE - Visualizar Relatório Final}                                                                 \\ \hline
ALR                           & DER's                                                & Total              \\ \hline
\multirow{13}{*}{Projeto}     & Descrição final do projeto             & 1 \\ \cline{2-3}
                             & Data de conclusão do projeto           & 1 \\ \cline{2-3}
                             & Desenho esquemático                    & 1 \\ \cline{2-3} 
			      & Subprojeto                             & 1 \\ \cline{2-3}
                             & Descrição final do subprojeto          & 1 \\ \cline{2-3}
                             & Benefícios alcançados                  & 1 \\ \cline{2-3}
                             & Data de conclusão do subprojeto        & 1 \\ \cline{2-3}
                             & Planilha de execução físico-financeira & 1 \\ \cline{2-3}
                             & Informações georreferenciadas          & 1 \\ \cline{2-3}
                             & Notas fiscais                          & 1 \\ \cline{2-3}
                             & Outros                                 & 1 \\ \cline{2-3}
                             & Item                                   & 1 \\ \cline{2-3}
                             & Nome                                   & 1 \\ \hline
DER's extras & Capacidade de iniciar ação             & 1 \\ \hline
\multicolumn{2}{|c|}{\textbf{TOTAL DE DER's}}                                                 & \textbf{14}                 \\ \hline
\end{tabular}
\end{table}

  \subsubsection{CE - Visualizar Portaria de habilitação}
  \begin{table}[!h]
\centering
\caption{CE - Visualizar Portaria de Habilitação}
\label{ce_visualizar_portaria_de_habilitacao}
\begin{tabular}{|c|l|c|}
\hline
\multicolumn{3}{|c|}{CE - Visualizar Portaria de Habilitação}                    \\ \hline
ALR                      & \multicolumn{1}{c|}{DER's} & Total \\ \hline
\multirow{4}{*}{Projeto} & Data da portaria                         & 1     \\ \cline{2-3} 
                         & Data de publicação            & 1     \\ \cline{2-3} 
                         & Nº da portaria      & 1     \\ \cline{2-3} 
                         & Portaria de habilitação              & 1     \\ \hline
DER's extras & Capacidade de iniciar ação             & 1 \\ \hline
\multicolumn{2}{|c|}{\textbf{TOTAL DE DER's}}                  & \textbf{5}     \\ \hline
\end{tabular}
\end{table}

      
	
   \subsubsection{CE - Selecionar Tipo de Relatório}
      

      \begin{table}[!h]
      \centering
      \caption{CE - Selecionar Tipo de Relatório}
      \label{ce_tipo_relatorio}
      \begin{tabular}{|c|l|c|}
      \hline
      \multicolumn{3}{|c|}{CE - Selecionar Tipo de Relatório}                              \\ \hline
      ALR                          & \multicolumn{1}{c|}{DER's}    & Total        \\ \hline
      Projeto                      & Tipo de Relatório             & 1            \\ \hline
      \multirow{2}{*}{DERs extras} & Capacidade de iniciar ação    & 1            \\ \cline{2-3} 
				  & Capacidade de emitir mensagem & 1            \\ \hline
       \multicolumn{2}{|c|}{\textbf{TOTAL DE DER's}}          & \textbf{3}         \\ \hline
      \end{tabular}
      \end{table}

      \vfill
      \pagebreak

  \subsubsection{CE - Visualizar histórico de relatórios}
\begin{table}[!h]
\centering
\caption{CE - Visualizar histórico de relatórios}
\label{ce_visualizar_historico}
\begin{tabular}{|l|l|l|}
\multicolumn{3}{c}{CE - Visualizar histórico de relatórios}          \\ \hline
ALR                      & DER's                      & Total         \\ \hline
\multirow{5}{*}{Projeto} & Relatório                  & 1             \\\cline{2-3}
                         & Data limite de submissão   & 1             \\\cline{2-3}
                         & Data de submissão          & 1             \\\cline{2-3}
                         & Ação                       & 1             \\\cline{2-3}
                         & Situação do relatório      & 1             \\\hline
DER's extras             & Capacidade de iniciar ação & 1             \\\hline
\multicolumn{2}{|c|}{\textbf{TOTAL DE DER's}}                     & \textbf{6} \\ \hline
\end{tabular}
\end{table}



  \subsubsection{CE - Listar projetos aprovados}
\begin{table}[!h]
\centering
\caption{CE - Listar projetos aprovados}
\label{ce_listar_projetos_aprovados}
\begin{tabular}{|l|l|l|}
\multicolumn{3}{c}{CE - Listar projetos aprovados}                   \\\hline
ALR                      & DER's                      & Total         \\\hline
\multirow{9}{*}{Projeto} & Campo de pesquisa          & 1             \\ \cline{2-3}
                         & Id do Projeto                         & 1             \\ \cline{2-3}
                         & Novo evento                & 1             \\\cline{2-3}
                         & Nome do Projeto                    & 1             \\\cline{2-3}
                         & Pj titular do projeto      & 1             \\\cline{2-3}
                         & Data de Submissão          & 1             \\\cline{2-3}
                         & Data de protocolo          & 1             \\\cline{2-3}
                         & Situação                   & 1             \\\cline{2-3}
                         & É consórcio                & 1             \\\hline
Usuário Interno          & Acompanhamento             & 1             \\\hline
DER's extras             & Capacidade de iniciar ação & 1             \\\hline
\multicolumn{2}{|c|}{\textbf{TOTAL DE DER's}}                   & \textbf{11}  \\ \hline
\end{tabular}
\end{table}

   \vfill
      \pagebreak
  \subsubsection{CE - Visualizar projetos}
  
      \begin{table}[!h]
      \centering
      \caption{CE - Visualizar projetos}
      \label{ce_visualizar_projetos}
      \begin{tabular}{|c|l|c|}
      \hline
      \multicolumn{3}{|c|}{CE - Visualizar Projetos}                                                              \\ \hline
      ALR                                                & \multicolumn{1}{c|}{DER's}   & Total              \\ \hline
      \multirow{7}{*}{Projeto}                           & ID do projeto                & \multirow{7}{*}{7} \\ \cline{2-2}
							& Nome do projeto              &                    \\ \cline{2-2}
							& Titular do projeto (PJ)      &                    \\ \cline{2-2}
							& Data de submissão            &                    \\ \cline{2-2}
							& Data de protocolo            &                    \\ \cline{2-2}
							& Situação                     &                    \\ \cline{2-2}
							& É consórcio                  &                    \\ \hline
      Usuário Interno                                   & Nome do analista responsável & 1                  \\ \hline
      \multicolumn{1}{|l|}{\multirow{2}{*}{DERs extras}} & Capacidade de iniciar ação   & \multirow{2}{*}{2} \\ \cline{2-2}
      \multicolumn{1}{|l|}{}                             & Filtro de pesquisa             &                    \\ \hline
      \multicolumn{2}{|c|}{\textbf{TOTAL DE DER's}}                                     & \textbf{10}        \\ \hline
      \end{tabular}
      \end{table}
      
   \subsubsection{CE - Detalhar projeto}
   
      \begin{table}[!h]
      \centering
      \caption{CE - Detalhar projeto}
      \label{ce_detalhar_projeto}
      \begin{tabular}{|c|l|c|}
      \hline
      \multicolumn{3}{|c|}{CE - Detalhar Projeto}                                  \\ \hline
      ALR                      & \multicolumn{1}{c|}{DER's}   & Total              \\ \hline
      \multirow{6}{*}{Projeto} & ID do projeto                & \multirow{6}{*}{6} \\ \cline{2-2}
			      & Situação                     &                    \\ \cline{2-2}
			      & Número da Versão             &                    \\ \cline{2-2}
			      & Data e hora                  &                    \\ \cline{2-2}
			      & Tipo de validação            &                    \\ \cline{2-2}
			      & Responsável pela versão      &                    \\ \hline
      Usuário Interno                 & Nome do analista responsável & 1                  \\ \hline
      DERs extras              & Capacidade de iniciar ação   & 1                  \\ \hline
      \multicolumn{2}{|c|}{\textbf{TOTAL DE DER's}}           & \textbf{8}         \\ \hline
      \end{tabular}
      \end{table}
      


      
      
    
    
\vfill
\pagebreak
  \subsection{Pontuação das funções de transação}
    
	\begin{table*}[!h]
	\centering
	\caption{Informações sobre as funções de transação}
	\label{funcoes_transacao}
	  \begin{tabular}{|p{0.40\linewidth}|p{0.20\linewidth}|p{0.14\linewidth}|}
	\hline
	\textbf{Tipo - Função} & \textbf{Complexidade} & \textbf{Pontos de função} \\
	\hline
	  EE - Suspender análise & Média & 4 \\ \hline
	  EE - Iniciar troca de mensagens & Média & 4 \\ \hline
	  EE - Enviar mensagens & Baixa & 3\\ \hline
	  EE - Encerrar troca de mensagens & Baixa & 3 \\ \hline
	  EE - Upload Portaria de Habilitação &Baixa & 3\\ \hline
	  EE - Submeter Relatório Parcial & Média & 4 \\ \hline
	  EE - Cadastrar Relatório Final & Média & 4  \\ \hline
	  EE - Alterar Relatório Final & Média & 4  \\ \hline
	  EE - Recusar Relatório & Baixa & 3\\ \hline
	  EE - Validar versão do projeto & Baixa &3 \\ \hline
	  EE - Recusar versão do projeto & Baixa & 3\\ \hline
	  EE - Alterar responsável & Baixa & 3\\ \hline
	  EE - Aceitar Relatorio & Baixa&3 \\ \hline
	  SE - Visualizar Relatório de Acompanhamento das Atividades do Projeto & Média & 5 \\ \hline
	  SE - Visualizar Relatório de Acompanhamento das Atividades por Analista & Baixa & 4\\ \hline
	  SE - Gerar Relatório: Base de Dados Geral & Alta& 7\\ \hline
	  SE - Gerar Relatório: Base de Acompanhamento & Média & 5\\ \hline
	  CE - Visualizar histórico de mensagens & Baixa& 3\\ \hline
	  CE - Listar projetos & Baixa& 3\\ \hline
	  CE - Visualizar Relatório Parcial & Média& 4\\ \hline
	  CE - Visualizar Relatório Final & Média&4 \\ \hline
	  CE - Visualizar Portaria de habilitação &Baixa & 3\\ \hline
	  CE - Listar projetos aprovados & Média& 4\\ \hline
	  CE - Visualizar histórico de relatórios & Baixa& 3\\ \hline
	  CE - Visualizar projetos & Média& 4\\ \hline
	  CE - Detalhar projeto & Média& 4\\ \hline
	  CE - Selecionar Tipo de Relatório &Baixa & 3\\ \hline
	  \multicolumn{2}{|l|}{\textbf{Total}} & \textbf{100} \\
	\hline
	\end{tabular}
	\end{table*}

\vfill
\pagebreak
\section{Resultado da contagem}

\begin{table*}[!h]
\centering
\caption{Pontos de Função}
\label{resultado_contagem}
  \begin{tabular}{|p{0.40\linewidth}|p{0.45\linewidth}|p{0.10\linewidth}|}
  \hline
  \textbf{Funções de Dados} & \textbf{Funções de Transação} & \textbf{Total} \\ 
  X & 100 & Y\\
  \hline
 
  \end{tabular}
\end{table*}

\section{Suposições e observações}

  \begin{itemize}
  
   \item No UC14, o passo alternativo A1 - "Solicitar informações para alterações de projeto" não foi contado porque representa a mesma funcionalidade
     do fluxo alternativo A1 do caso de uso UC15, que foi contabilizado.
   
   \item Para a função 'Upload Portaria de Habilitação' foi suposto que os dados são salvos no ALI Projeto;
   
   \item A funcionalidade de 'Exportar PDF' não foi contada por ser apenas uma outra forma de visualização do Relatório, a qual já foi contado
   no processo 'Visualizar Relatório Final';
   
   \item A funcionalidade de 'Download portaria de habilitação' não foi contada por ser apenas uma outra forma de visualização da portaria, a qual
   já foi contada no processo 'Visualizar Portaria de habilitação';
   
   \item A funcionalidade 'Submeter Relatório Final' por ser compatível com a funcionalidade 'Cadastrar Relatorio final'
   que já foi contada;
   
   \item Assumiu-se que o ID do projeto tem valor significativo para o usuário, pois no protótipo de tela contido na documentação o ID não 
    parecia ser um número aleatório. Tal assunção impactou na contagem do ALI Projeto;
   
   \item O fluxo alternativo A1 do UC21, responsável por emitir o relatório de relação de projetos, não foi contado por ser apenas outra forma
    de visualização dos dados dos projetos, já contado no processo elementar CE - Visualizar Projetos (Tabela \ref{ce_visualizar_projetos});
   
   \item Para a contagem do processo elementar SE - Visualizar Relatório de Acompanhamento das Atividades do Projeto
    (Tabela \ref{se_visualizar_relatorio_acompanhamento}) considerou-se que a data de referência para o filtro seria a Data de submissão;
    
   \item A funcionalidade "Gerar PDF" no caso de uso UC20 não foi contada por ser apenas uma outra forma de visualização do Relatório gerado
     pelos processos elementares SE - Visualizar Relatório de Acompanhamento das Atividades por Analista e 
     SE - Visualizar Relatório de Acompanhamento das Atividades do Projeto
     (Tabelas \ref{se_visualizar_relatorio_acompanhamento_analista} e \ref{se_visualizar_relatorio_acompanhamento});
     
   \item Para a contagem do processo elementar CE - Detalhar projeto (Tabela \ref{ce_detalhar_projeto}), por falta de informações na documentação,
     foi considerado que ao acionar este processo serão apresentados apenas os dados da versão do projeto mais o ID do projeto, e que não há 
     mensagens de erros;
     
   \item Os processos elementares EE - Validar versão do projeto e EE - Recusar versão do projeto 
    (Tabelas \ref{ee_validar_versao_projeto} e \ref{ee_resusar_versao_projeto}) foram descritos superficialmente por falta de informações
    na documentação disponível, considerando apenas que os processos atualizam a versão do projeto, a data e hora da versão e o tipo 
    de validação. A situação do projeto é usada com filtro de pesquisa;
    
   \item Para a contegem do processo elementar EE - Alterar responsável, considerou-se que o referido 'responsável' é a pessoa
   responsável pelo projeto. Portanto, seu ARL é Projeto (Tabela \ref{ali_projeto}). Além disso, considerou-se que como DER se tem o 
   nome do responsável.
   
   \item A funcionalidade de 'Imprimir histórico de mensagens' do UC15 não conta porque é só outra forma de apresentação.
  \end{itemize}

