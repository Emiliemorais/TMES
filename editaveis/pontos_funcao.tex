\section{Documentação utilizada}

\begin{itemize}
 \item Especificações de Caso de Uso;
 \item Regras de negócio;
 \item Proposta de Especificação do Software;
 \item Protótipos de telas;
\end{itemize}

\section{Participantes, seus papéis e qualificações}

\section{Data}
26/09/2015

\section{Funções de Dados}

Foram identificados os seguintes Arquivos Lógicos Internos (ALI):

\begin{table}[]
\centering
\caption{ALI - Projeto}
\label{ali_projeto}
\begin{tabular}{|c|c|l|}
\hline
\multicolumn{3}{|c|}{\textbf{ALI - Projeto}}                                                      \\ \hline
\multicolumn{1}{|l|}{RLR's} & Tipo                         & \multicolumn{1}{c|}{DER's}  \\ \hline
\multirow{8}{*}{1}          & \multirow{8}{*}{Obrigatório} & ID do projeto               \\ \cline{3-3} 
                            &                              & Nome do projeto             \\ \cline{3-3} 
                            &                              & Titular do projeto (PJ)     \\ \cline{3-3} 
                            &                              & Data de submissão           \\ \cline{3-3} 
                            &                              & Data de protocolo           \\ \cline{3-3} 
                            &                              & Situação                    \\ \cline{3-3} 
                            &                              & É consórcio                 \\ \cline{3-3} 
                            &                              & Responsável pelo projeto    \\ \hline
\multirow{4}{*}{2}        & \multirow{4}{*}{Obrigatório} & Número da versão do projeto \\ \cline{3-3} 
                            &                              & Data e hora da versão       \\ \cline{3-3} 
                            &                              & Tipo de validação da versão \\ \cline{3-3} 
                            &                              & Responsável pela versão     \\ \hline
\end{tabular}
\end{table}


\begin{table*}[!h]
\centering
\caption{Informações sobre as funções de dados}
\label{funcoes_dados}
  \begin{tabular}{|p{0.12\linewidth}|p{0.10\linewidth}|p{0.14\linewidth}|p{0.14\linewidth}|p{0.17\linewidth}|p{0.10\linewidth}|}
  \hline
  \textbf{Função} & \textbf{Tipo} & \textbf{Quantidade de RLR} & \textbf{Quantidade de DER} & \textbf{Complexidade} & \textbf{Pontos de função (PF)} \\
    \hline
  Usuário & ALI & 1 &  &  & \\
  \hline
  Projeto & ALI & 2 & 12 & Baixa & 5 \\
  \hline
  Subprojeto & ALI & &  & & \\
  \hline
  Histórico do projeto & ALI & & & &\\
  \hline
  Relatório & ALI & &  & & \\
  \hline
  \end{tabular}
\end{table*}

\vfill
\pagebreak
\section{Funções de Transação}

\begin{table*}[!h]
\centering
\caption{Informações sobre as funções de transação}
\label{funcoes_transacao}
  \begin{tabular}{p{0.20\linewidth}p{0.10\linewidth}p{0.14\linewidth}p{0.14\linewidth}p{0.17\linewidth}p{0.10\linewidth}}
\hline
\textbf{Função} & \textbf{Tipo} & \textbf{Quantidade de ALR} & \textbf{Quantidade de DER} & \textbf{Complexidade} & \textbf{Pontos de função} \\
\hline
Listar projetos & CE & 1 & 5 & Baixa & 3 \\
\hline
Upload Portaria de Habilitação & EE & 1 & 6 & Baixa & 3 \\
\hline
Submeter Relatório Parcial & EE & 1 & 11 & Baixa & 3 \\
\hline
Visualizar Relatório Parcial & CE & 1 & 7 & Baixa & 3 \\
\hline
Cadastrar Relatório Final & EE & 2 & 14 & Média & 4 \\
\hline
Alterar Relatório Final & EE & 2 & 14 & Média & 4 \\
\hline
Visualizar Relatório Final & CE & 1 & 13 & Baixa & 3 \\
\hline
Submeter Relatório Final &  &  &  &  &  \\
\hline
Visualizar Portaria de habilitação & CE & 1 & 5 & Baixa & 3 \\
\hline
Listar projetos aprovados & CE & 1 & 11 & Baixa & 3 \\
\hline
Visualizar informações de habilitação e histórico de relatórios & CE & 1 & 6 & Baixa & 3 \\
\hline
Aceitar Relatorio Parcial & EE & 1 & 3 & Baixa & 3 \\
\hline
Recusar Relatório parcial & EE & 1 & 3 & Baixa & 3\\
\end{tabular}
\end{table*}

\vfill
\pagebreak
\label{der_alr_info}
\begin{longtable}{p{0.20\linewidth}p{0.20\linewidth}p{0.50\linewidth}}
  \hline
  \textbf{Função} & \textbf{ALRs} & \textbf{DERs} \\
   \hline
    Listar projetos&	Projeto&	ID, Nome do projeto, PJ titular do projeto, Versão válida, Situação do projeto e Capacidade de iniciar ação\\

    \hline
    Upload Portaria de Habilitação&	    Projeto &	Data da portaria, Data de publicação, Nº da portaria, Portaria de habilitação, Capacidade de iniciar ação e Capacidade de emitir mensagem \\

    \hline
    Submeter Relatório Parcial&	Relatório&	Houve execução física e/ou financeira do projeto?, Não alterado, Desenho esquemático, Subprojeto, Houve execução física e/ou financeira do
    Subprojeto?,Planilha de execução físico-financeira, Informações georreferenciadas, Outros, Projeto como executado, Capacidade de iniciar ação e Capacidade de emitir mensagem\\
    \hline
    Visualizar Relatório Parcial&	Relatório&	Desenho esquemático, Subprojeto, Planilha de execução físico-financeira, Informações georreferenciadas, Outros, Projeto como executado, Capacidade de 
    iniciar ação\\
    \hline
    Cadastrar Relatório Final e Alterar Relatório Final&	Relatório e Subprojeto&	Descrição final do projeto, Data de conclusão do projeto, Desenho esquemático, Subprojeto, Descrição final do subprojeto, Benefícios alcançados,
    Data de conclusão do subprojeto, Planilha de execução físico-financeira, Informações georreferenciadas, Notas fiscais, Outros, Item, Nome, Capacidade de iniciar ação e Capacidade de emitir mensagem\\

    \hline
    Visualizar Relatório Parcial&	Relatório&	Descrição final do projeto, Data de conclusão do projeto, Desenho esquemático, Subprojeto, Descrição final do subprojeto, Benefícios alcançados, Data de 
    conclusão do subprojeto, Planilha de execução físico-financeira, Informações georreferenciadas, Notas fiscais, Outros, Item, Nome, Capacidade de iniciar ação \\
    \hline
    Visualizar Portaria de habilitação&	    Projeto &	Data da portaria, Data de publicação, Nº da portaria, Portaria de habilitação e Capacidade de iniciar ação\\

    \hline
    Listar projetos aprovados&	Projeto&	Campo de pesquisa, id,novo evento,  projeto, acompanhamento, Pj titular do projeto, Data de Submissão, Data de protocolo, Situação, É consórcio e Capacidade de iniciar ação\\
    \hline
    Visualizar informações de habilitação e histórico de relatórios&	Relatório&	"Relatório, Data limite de submissão, Data de submissão, Ação, Situação e Capacidade de iniciar ação"\\

    \hline
    Aceitar Relatorio Parcial&	Relatório&	Situação, Capacidade de iniciar ação e capacidade de emitir mensagem\\

    \hline
    Recusar Relatório parcial&	Relatório&	Situação, Capacidade de iniciar ação e capacidade de emitir mensagem\\
    \hline
\caption{Informações sobre os DERs e os ALRs}
 \end{longtable}


\pagebreak
\section{Resultado da contagem}

\begin{table*}[!h]
\centering
\caption{Pontos de Função}
\label{resultado_contagem}
  \begin{tabular}{|p{0.20\linewidth}|p{0.25\linewidth}|p{0.20\linewidth}|}
  \hline
  \textbf{Funções de Dados} & \textbf{Funções de Transação} & \textbf{Total} \\ 
  \hline
 
  \end{tabular}
\end{table*}

\pagebreak
\section{Suposições e observações}

  \begin{itemize}
   \item Para a função 'Upload Portaria de Habilitação' foi suposto que os dados são salvos no ALI Projeto;
   \item A funcionalidade de 'Exportar PDF' não foi contada por ser apenas uma outra forma de visualização do Relatório, a qual já foi contado
   no processo 'Visualizar Relatório Final';
   \item A funcionalidade de 'Download portaria de habilitação' não foi contada por ser apenas uma outra forma de visualização da portaria, a qual
   já foi contada no processo 'Visualizar Portaria de habilitação';
   \item Assumiu-se que o ID do projeto tem valor significativo para o usuário, pois no protótipo de tela contido na documentação o ID não 
    parecia ser um número aleatório. Tal assunção impactou na contagem do ALI Projeto;
  \end{itemize}

