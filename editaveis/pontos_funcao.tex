\section{Documentação utilizada}

\begin{itemize}
 \item Especificações de Caso de Uso;
 \item Regras de negócio;
 \item Proposta de Especificação do Software;
 \item Protótipos de telas;
\end{itemize}

\section{Participantes, seus papéis e qualificações}

\section{Data}
26/09/2015

\vfill
\pagebreak
\section{Funções de Dados}

Esta seção apresenta os Arquivos Lógicos Internos (ALI) e Arquivos de Interface Externa (AIE) levantados, bem como a 
pontuação dos mesmos na contagem realizada.

  \subsection{ALI - Projeto}
  
      \begin{table}[!h]
      \centering
      \caption{ALI - Projeto}
      \label{ali_projeto}
      \begin{tabular}{|c|c|l|}
      \hline
      \multicolumn{3}{|c|}{\textbf{ALI - Projeto}}                                                      \\ \hline
      \multicolumn{1}{|l|}{RLR's} & Tipo                         & \multicolumn{1}{c|}{DER's}  \\ \hline
      \multirow{8}{*}{1}          & \multirow{8}{*}{Obrigatório} & ID do projeto               \\ \cline{3-3} 
				  &                              & Nome do projeto             \\ \cline{3-3} 
				  &                              & Titular do projeto (PJ)     \\ \cline{3-3} 
				  &                              & Data de submissão           \\ \cline{3-3} 
				  &                              & Data de protocolo           \\ \cline{3-3} 
				  &                              & Situação                    \\ \cline{3-3} 
				  &                              & É consórcio                 \\ \cline{3-3} 
				  &                              & Responsável pelo projeto    \\ \hline
      \multirow{4}{*}{2}        & \multirow{4}{*}{Obrigatório} & Número da versão do projeto \\ \cline{3-3} 
				  &                              & Data e hora da versão       \\ \cline{3-3} 
				  &                              & Tipo de validação da versão \\ \cline{3-3} 
				  &                              & Responsável pela versão     \\ \hline
      \end{tabular}
      \end{table}

  \subsection{Pontuação das funções de dados}
  
      \begin{table*}[!h]
      \centering
      \caption{Informações sobre as funções de dados}
      \label{funcoes_dados}
	\begin{tabular}{|p{0.12\linewidth}|p{0.10\linewidth}|p{0.14\linewidth}|p{0.14\linewidth}|p{0.17\linewidth}|p{0.10\linewidth}|}
	\hline
	\textbf{Função} & \textbf{Tipo} & \textbf{Quantidade de RLR} & \textbf{Quantidade de DER} & \textbf{Complexidade} & \textbf{Pontos de função (PF)} \\
	  \hline
	Usuário & ALI & 1 &  &  & \\
	\hline
	Projeto & ALI & 2 & 12 & Baixa & 5 \\
	\hline
	Subprojeto & ALI & &  & & \\
	\hline
	Relatório & ALI & &  & & \\
	\hline
	\end{tabular}
      \end{table*}

\vfill
\pagebreak
\section{Funções de Transação}
  
  Esta seção apresenta as funções de transação identificadas, bem como suas classificações e pontuação na contagem realizada.
  
  \subsection{CE - Listar projetos}

\begin{table}[!h]
\centering
\caption{CE - Listar projetos}
\label{ce_listar_projeto}
\begin{tabular}{|c|l|c|}
\hline
\multicolumn{3}{|c|}{CE - Listar projetos}                    \\ \hline
ALR                      & \multicolumn{1}{c|}{DER's} & Total \\ \hline
\multirow{5}{*}{Projeto} & ID                         & 1     \\ \cline{2-3} 
                         & Nome do projeto            & 1     \\ \cline{2-3} 
                         & PJ titular do projeto      & 1     \\ \cline{2-3} 
                         & Versão válida              & 1     \\ \cline{2-3} 
                         & Situação do projeto        & 1     \\ \hline
DER Extra                & Capacidade de iniciar ação & 1     \\ \hline
\multicolumn{2}{|c|}{\textbf{TOTAL DE DER's}}                  & 6     \\ \hline
\end{tabular}
\end{table}

  \subsection{EE - Upload Portaria de Habilitação}
\begin{table}[!h]
\centering
\caption{EE - Upload Portaria de Habilitação}
\label{ee_upload_portaria_de_habilitacao}
\begin{tabular}{|c|l|c|}
\hline
\multicolumn{3}{|c|}{CE - Listar projetos}                    \\ \hline
ALR                      & \multicolumn{1}{c|}{DER's} & Total \\ \hline
\multirow{5}{*}{Projeto} & ID                         & 1     \\ \cline{2-3} 
                         & Nome do projeto            & 1     \\ \cline{2-3} 
                         & PJ titular do projeto      & 1     \\ \cline{2-3} 
                         & Versão válida              & 1     \\ \cline{2-3} 
                         & Situação do projeto        & 1     \\ \hline
DER's extras                & Capacidade de iniciar ação & 1     \\ \hline
\multicolumn{2}{|c|}{\textbf{TOTAL DE DER's}}                  & \textbf{6}     \\ \hline
\end{tabular}
\end{table}
  \subsection{EE - Submeter Relatório Parcial}

\begin{table}[!h]
\centering
\caption{EE - Submeter Relatório Parcial}
\label{ee_submeter_relatorio_parcial}
\begin{tabular}{|l|l|l|}
\hline
\multicolumn{3}{|c|}{EE - Submeter Relatório Parcial}                                                                 \\ \hline
ALR                           & DER's                                                & Total              \\ \hline
\multirow{9}{*}{Relatório}    & Houve execução física e/ou financeira do projeto?    & 1                  \\ \cline{2-3} 
                              & Não alterado                                         & 1                  \\ \cline{2-3} 
                              & Desenho esquemático                                  & 1                  \\ \cline{2-3} 
                              & Subprojeto                                           & 1                  \\ \cline{2-3} 
                              & Houve execução física e/ou financeira do Subprojeto? & 1                  \\ \cline{2-3} 
                              & Planilha de execução físico-financeira               & 1                  \\ \cline{2-3} 
                              & Informações georreferenciadas                        & 1                  \\ \cline{2-3} 
                              & Outros                                               & 1                  \\ \cline{2-3} 
                              & Projeto como executado                               & 1                  \\ \hline
\multirow{2}{*}{DER's extras} & Capacidade de iniciar ação                           & \multirow{2}{*}{2} \\ \cline{2-2}
                              & Capacidade de emitir mensagem                        &                    \\ \hline
\multicolumn{2}{|c|}{\textbf{TOTAL DE DER's}}                                                 & \textbf{11}                 \\ \hline
\end{tabular}
\end{table}


  \subsection{CE - Visualizar Relatório Parcial}
  
  \begin{table}[!h]
\centering
\caption{CE - Visualizar Relatório Parcial}
\label{ce_visualizar_relatorio_parcial}
\begin{tabular}{|l|l|l|}
\hline
\multicolumn{3}{|c|}{CE - Visualizar Relatório Parcial}                                                                 \\ \hline
ALR                           & DER's                                                & Total              \\ \hline
\multirow{9}{*}{Relatório}    & Desenho esquemático                                  & 1                  \\ \cline{2-3}  
                              & Subprojeto                                           & 1                  \\ \cline{2-3} 
                              & Planilha de execução físico-financeira               & 1                  \\ \cline{2-3} 
                              & Informações georreferenciadas                        & 1                  \\ \cline{2-3} 
                              & Outros                                               & 1                  \\ \cline{2-3} 
                              & Projeto como executado                               & 1                  \\ \hline
DER's extras		      & Capacidade de iniciar ação                           & 1 \\ \hline
\multicolumn{2}{|c|}{\textbf{TOTAL DE DER's}}                                                 & \textbf{7}                 \\ \hline
\end{tabular}
\end{table}


  \subsection{EE - Cadastrar Relatório Final}
  
\begin{table}[]
\centering
\caption{EE - Cadastrar Relatório Final}
\label{ee_cadastrar_relatório_final}
\begin{tabular}{|l|l|l|}
\hline
\multicolumn{3}{|c|}{EE - Cadastrar Relatório Final}                                                                 \\ \hline
ALR                           & DER's                                                & Total              \\ \hline
\multirow{3}{*}{Projeto}     & Descrição final do projeto             & 1 \\ \cline{2-3}
                             & Data de conclusão do projeto           & 1 \\ \cline{2-3}
                             & Desenho esquemático                    & 1 \\ \cline{2-3} \hline
\multirow{10}{*}{Subprojeto} & Subprojeto                             & 1 \\ \cline{2-3}
                             & Descrição final do subprojeto          & 1 \\ \cline{2-3}
                             & Benefícios alcançados                  & 1 \\ \cline{2-3}
                             & Data de conclusão do subprojeto        & 1 \\ \cline{2-3}
                             & Planilha de execução físico-financeira & 1 \\ \cline{2-3}
                             & Informações georreferenciadas          & 1 \\ \cline{2-3}
                             & Notas fiscais                          & 1 \\ \cline{2-3}
                             & Outros                                 & 1 \\ \cline{2-3}
                             & Item                                   & 1 \\ \cline{2-3} \hline
                             & Nome                                   & 1 \\ \cline{2-3}
DER's extras                 & Capacidade de iniciar ação             & 1 \\ \hline
\multicolumn{2}{|c|}{\textbf{TOTAL DE DER's}}                                                 & \textbf{14}                 \\ \hline
\end{tabular}
\end{table}
  
  \subsection{EE - Alterar Relatório Final}
  \subsection{CE - Visualizar Relatório Final}
  \subsection{CE - Visualizar Portaria de habilitação}
  \subsection{CE - Listar projetos aprovados}
  \subsection{CE - Visualizar informações de habilitação e histórico de relatórios}
  \subsection{EE - Aceitar Relatorio Parcial}
  \subsection{EE - Recusar Relatório parcial}
  
  \subsection{SE - Visualizar Relatório de Acompanhamento das Atividades do Projeto}
  
    Este processo elementar foi classificado como Saída Externa porque, segundo a Regra de Negócio RN63, neste relatório
    deve conter a quantidade de projetos por situação, que é calculada na hora de gerar o relatório.
    
      \begin{table}[!h]
      \centering
      \caption{SE - Visualizar Relatório de Acompanhamento das Atividades do Projeto}
      \label{se_visualizar_relatorio_acompanhamento}
      \begin{tabular}{|c|l|c|}
      \hline
      \multicolumn{3}{|c|}{SE - Visualizar Relatório de Acompanhamento das Atividades do Projeto} \\ \hline
      ALR                              & \multicolumn{1}{c|}{DER's}       & Total                 \\ \hline
      \multirow{2}{*}{Projeto}         & Situação                         & 1                     \\ \cline{2-3} 
				      & Data de submissão                & 1                     \\ \hline
      Usuário                          & Nome do analista responsável     & 1                     \\ \hline
      \multirow{4}{*}{DERs extras}     & Capacidade de iniciar ação       & \multirow{4}{*}{4}    \\ \cline{2-2}
				      & Mensagem de erro                 &                       \\ \cline{2-2}
				      & Data inicial                     &                       \\ \cline{2-2}
				      & Data final                       &                       \\ \hline
      \multicolumn{2}{|c|}{\textbf{TOTAL DE DER's}}                       & \textbf{7}            \\ \hline
      \end{tabular}
      \end{table}
      
  \subsection{SE - Visualizar Relatório de Acompanhamento das Atividades por Analista}
      
      Este processo elementar foi classificado como Saída Externa porque, segundo a Regra de Negócio RN63, neste relatório
    deve conter a quantidade de projetos por situação, que é calculada na hora de gerar o relatório.
      
      \begin{table}[!h]
      \centering
      \caption{SE - Visualizar Relatório de Acompanhamento das Atividades por Analista}
      \label{se_visualizar_relatorio_acompanhamento_analista}
      \begin{tabular}{|c|l|c|}
      \hline
      \multicolumn{3}{|c|}{SE - Visualizar Relatório de Acompanhamento das Atividades por Analista} \\ \hline
      ALR                               & \multicolumn{1}{c|}{DER's}       & Total                  \\ \hline
      Projeto                           & Situação                         & 1                      \\ \hline
      Usuário                           & Nome do analista responsável     & 1                      \\ \hline
      \multirow{3}{*}{DERs extras}      & Capacidade de iniciar ação       & \multirow{3}{*}{3}     \\ \cline{2-2}
					& Mensagem de erro                 &                        \\ \cline{2-2}
					& Total de projetos                &                        \\ \hline
      \textbf{TOTAL DE DER's}           &                                  & \textbf{5}             \\ \hline
      \end{tabular}
      \end{table}
  
  \vfill
  \pagebreak
  \subsection{CE - Listar projeto}
  
      \begin{table}[!h]
      \centering
      \caption{CE - Listar projeto}
      \label{ce_listar_projeto}
      \begin{tabular}{|c|l|c|}
      \hline
      \multicolumn{3}{|c|}{CE - Listar Projeto}                                                              \\ \hline
      ALR                                                & \multicolumn{1}{c|}{DER's}   & Total              \\ \hline
      \multirow{7}{*}{Projeto}                           & ID do projeto                & \multirow{7}{*}{7} \\ \cline{2-2}
							& Nome do projeto              &                    \\ \cline{2-2}
							& Titular do projeto (PJ)      &                    \\ \cline{2-2}
							& Data de submissão            &                    \\ \cline{2-2}
							& Data de protocolo            &                    \\ \cline{2-2}
							& Situação                     &                    \\ \cline{2-2}
							& É consórcio                  &                    \\ \hline
      Usuário                                            & Nome do analista responsável & 1                  \\ \hline
      \multicolumn{1}{|l|}{\multirow{2}{*}{DERs extras}} & Capacidade de iniciar ação   & \multirow{2}{*}{2} \\ \cline{2-2}
      \multicolumn{1}{|l|}{}                             & Mensagem de erro             &                    \\ \hline
      \multicolumn{2}{|c|}{\textbf{TOTAL DE DER's}}                                     & \textbf{10}        \\ \hline
      \end{tabular}
      \end{table}
  
  \subsection{CE - Emitir Relatório de Relação de Projeto}
    
      \begin{table}[!h]
	\centering
	\caption{CE - Emitir Relatório de Relação de Projeto}
	\label{ce_emitir_relatorio_projeto}
	\begin{tabular}{|c|l|c|}
	\hline
	\multicolumn{3}{|c|}{CE - Emitir Relatório de Relação de Projeto}                                      \\ \hline
	ALR                                                & \multicolumn{1}{c|}{DER's}   & Total              \\ \hline
	\multirow{6}{*}{Projeto}                           & ID do projeto                & \multirow{6}{*}{6} \\ \cline{2-2}
							  & Nome do projeto              &                    \\ \cline{2-2}
							  & Titular do projeto (PJ)      &                    \\ \cline{2-2}
							  & Data de submissão            &                    \\ \cline{2-2}
							  & Data de protocolo            &                    \\ \cline{2-2}
							  & Situação                     &                    \\ \hline
	Usuário                                            & Nome do analista responsável & 1                  \\ \hline
	\multicolumn{1}{|l|}{\multirow{2}{*}{DERs extras}} & Capacidade de iniciar ação   & \multirow{2}{*}{2} \\ \cline{2-2}
	\multicolumn{1}{|l|}{}                             & Mensagem de erro             &                    \\ \hline
	\multicolumn{2}{|c|}{\textbf{TOTAL DE DER's}}                                     & \textbf{9}         \\ \hline
	\end{tabular}
	\end{table}
  
    
  
   \subsection{Pontuação das funções de transação}
   
	\begin{table*}[!h]
	\centering
	\caption{Informações sobre as funções de transação}
	\label{funcoes_transacao}
	  \begin{tabular}{p{0.20\linewidth}p{0.10\linewidth}p{0.14\linewidth}p{0.14\linewidth}p{0.17\linewidth}p{0.10\linewidth}}
	\hline
	\textbf{Função} & \textbf{Tipo} & \textbf{Quantidade de ALR} & \textbf{Quantidade de DER} & \textbf{Complexidade} & \textbf{Pontos de função} \\
	\hline
	Listar projetos & CE & 1 & 5 & Baixa & 3 \\
	\hline
	Upload Portaria de Habilitação & EE & 1 & 6 & Baixa & 3 \\
	\hline
	Submeter Relatório Parcial & EE & 1 & 11 & Baixa & 3 \\
	\hline
	Visualizar Relatório Parcial & CE & 1 & 7 & Baixa & 3 \\
	\hline
	Cadastrar Relatório Final & EE & 2 & 14 & Média & 4 \\
	\hline
	Alterar Relatório Final & EE & 2 & 14 & Média & 4 \\
	\hline
	Visualizar Relatório Final & CE & 1 & 13 & Baixa & 3 \\
	\hline
	Submeter Relatório Final &  &  &  &  &  \\
	\hline
	Visualizar Portaria de habilitação & CE & 1 & 5 & Baixa & 3 \\
	\hline
	Listar projetos aprovados & CE & 1 & 11 & Baixa & 3 \\
	\hline
	Visualizar informações de habilitação e histórico de relatórios & CE & 1 & 6 & Baixa & 3 \\
	\hline
	Aceitar Relatorio Parcial & EE & 1 & 3 & Baixa & 3 \\
	\hline
	Recusar Relatório parcial & EE & 1 & 3 & Baixa & 3\\
	\end{tabular}
	\end{table*}

	\vfill
	\pagebreak
	\label{der_alr_info}
	\begin{longtable}{p{0.20\linewidth}p{0.20\linewidth}p{0.50\linewidth}}
	  \hline
	  \textbf{Função} & \textbf{ALRs} & \textbf{DERs} \\
	  \hline
	    Listar projetos&	Projeto&	ID, Nome do projeto, PJ titular do projeto, Versão válida, Situação do projeto e Capacidade de iniciar ação\\

	    \hline
	    Upload Portaria de Habilitação&	    Projeto &	Data da portaria, Data de publicação, Nº da portaria, Portaria de habilitação, Capacidade de iniciar ação e Capacidade de emitir mensagem \\

	    \hline
	    Submeter Relatório Parcial&	Relatório&	Houve execução física e/ou financeira do projeto?, Não alterado, Desenho esquemático, Subprojeto, Houve execução física e/ou financeira do
	    Subprojeto?,Planilha de execução físico-financeira, Informações georreferenciadas, Outros, Projeto como executado, Capacidade de iniciar ação e Capacidade de emitir mensagem\\
	    \hline
	    Visualizar Relatório Parcial&	Relatório&	Desenho esquemático, Subprojeto, Planilha de execução físico-financeira, Informações georreferenciadas, Outros, Projeto como executado, Capacidade de 
	    iniciar ação\\
	    \hline
	    Cadastrar Relatório Final e Alterar Relatório Final&	Relatório e Subprojeto&	Descrição final do projeto, Data de conclusão do projeto, Desenho esquemático, Subprojeto, Descrição final do subprojeto, Benefícios alcançados,
	    Data de conclusão do subprojeto, Planilha de execução físico-financeira, Informações georreferenciadas, Notas fiscais, Outros, Item, Nome, Capacidade de iniciar ação e Capacidade de emitir mensagem\\

	    \hline
	    Visualizar Relatório Parcial&	Relatório&	Descrição final do projeto, Data de conclusão do projeto, Desenho esquemático, Subprojeto, Descrição final do subprojeto, Benefícios alcançados, Data de 
	    conclusão do subprojeto, Planilha de execução físico-financeira, Informações georreferenciadas, Notas fiscais, Outros, Item, Nome, Capacidade de iniciar ação \\
	    \hline
	    Visualizar Portaria de habilitação&	    Projeto &	Data da portaria, Data de publicação, Nº da portaria, Portaria de habilitação e Capacidade de iniciar ação\\

	    \hline
	    Listar projetos aprovados&	Projeto&	Campo de pesquisa, id,novo evento,  projeto, acompanhamento, Pj titular do projeto, Data de Submissão, Data de protocolo, Situação, É consórcio e Capacidade de iniciar ação\\
	    \hline
	    Visualizar informações de habilitação e histórico de relatórios&	Relatório&	"Relatório, Data limite de submissão, Data de submissão, Ação, Situação e Capacidade de iniciar ação"\\

	    \hline
	    Aceitar Relatorio Parcial&	Relatório&	Situação, Capacidade de iniciar ação e capacidade de emitir mensagem\\

	    \hline
	    Recusar Relatório parcial&	Relatório&	Situação, Capacidade de iniciar ação e capacidade de emitir mensagem\\
	    \hline
	\caption{Informações sobre os DERs e os ALRs}
	\end{longtable}

\vfill
\pagebreak
\section{Resultado da contagem}

\begin{table*}[!h]
\centering
\caption{Pontos de Função}
\label{resultado_contagem}
  \begin{tabular}{|p{0.20\linewidth}|p{0.25\linewidth}|p{0.20\linewidth}|}
  \hline
  \textbf{Funções de Dados} & \textbf{Funções de Transação} & \textbf{Total} \\ 
  \hline
 
  \end{tabular}
\end{table*}

\vfill
\pagebreak
\section{Suposições e observações}

  \begin{itemize}
   \item Para a função 'Upload Portaria de Habilitação' foi suposto que os dados são salvos no ALI Projeto;
   
   \item A funcionalidade de 'Exportar PDF' não foi contada por ser apenas uma outra forma de visualização do Relatório, a qual já foi contado
   no processo 'Visualizar Relatório Final';
   
   \item A funcionalidade de 'Download portaria de habilitação' não foi contada por ser apenas uma outra forma de visualização da portaria, a qual
   já foi contada no processo 'Visualizar Portaria de habilitação';
   
   \item Assumiu-se que o ID do projeto tem valor significativo para o usuário, pois no protótipo de tela contido na documentação o ID não 
    parecia ser um número aleatório. Tal assunção impactou na contagem do ALI Projeto;
   
   \item Para a contagem do processo elementar SE - Visualizar Relatório de Acompanhamento das Atividades do Projeto
    (Tabela \ref{se_visualizar_relatorio_acompanhamento}) considerou-se que a data de referência para o filtro seria a Data de submissão;
    
   \item A funcionalidade "Gerar PDF" no caso de uso UC20 não foi contada por ser apenas uma outra forma de visualização do Relatório gerado
     pelos processos elementares SE - Visualizar Relatório de Acompanhamento das Atividades por Analista e 
     SE - Visualizar Relatório de Acompanhamento das Atividades do Projeto
     (Tabelas \ref{se_visualizar_relatorio_acompanhamento_analista} e \ref{se_visualizar_relatorio_acompanhamento});
  \end{itemize}

