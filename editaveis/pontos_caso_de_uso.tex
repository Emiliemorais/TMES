\section{Propósito e tipo da contagem}

\section{Fontes de informação utilizadas para identificar os RFU usados na medição}
As informações utilizadas para a contagem foram retiradas das especificações de caso de uso do sistema.

\section{Participantes, seus papéis e qualificações}

\section{Data}
25/09/2015

\section{Atores}

\begin{table*}[!h]
\centering
\caption{Atores do sistema}
\label{Rotulo}
  \begin{tabular}{|p{0.20\linewidth}|p{0.25\linewidth}|p{0.30\linewidth}|p{0.10\linewidth}|}
  \hline
  \textbf{Nome do ator} & \textbf{Complexidade} & \textbf{Justificativa} & \textbf{PCU} \\ 
  \hline

  Analista & Alta & Usuário humano que interage com o sistema através de interface gráfica ou página Web& 3 \\
  \hline
  Proponente & Alta & Usuário humano que interage com o sistema através de interface gráfica ou página Web& 3\\
  \hline
  Supervisor & Alta & Usuário humano que interage com o sistema através de interface gráfica ou página Web& 3\\
  \hline
  \end{tabular}
\end{table*}

\section{Casos de Uso}
  
  Esta seção apresenta a contagem das transações realizadas para cada caso de uso definido no escopo de contagem.

  \pagebreak
\subsection{UC14 - Ferramentas Administrativas}
   
  Este caso de uso descreve a funcionalidade de suspender análise do projeto e solicitar informações para alterações de projetos.
  
  \subsubsection{Transações}

  \begin{table*}[!h]
  \centering
  \caption{Transações do UC14}
  \label{uc14_transactions}
    \begin{tabular}{|p{0.20\linewidth}|p{0.25\linewidth}|p{0.20\linewidth}|}
    \hline
    \textbf{Transação} & \textbf{Fluxo} & \textbf{Passos} \\ 
    \hline
    1 & Básico & P1\\
    \hline
    2 & Básico & P2 e P3\\
    \hline
    3 & Básico & P4 e P5\\
    \hline
    4 & Alternativo & A1 e A2\\
    \hline
    5 & Alternativo & A3 e A4\\
    \hline
    6 & Alternativo & A5 e A6\\
    \hline
    \end{tabular}
  \end{table*}

  \subsubsection{Pontos de Caso de Uso}

  \begin{itemize}
  \item Quantidade de transações: 6;
  \item Complexidade: Média;
  \item Total: 10.
  \end{itemize}

  \vfill

\pagebreak
\subsection{UC15 - Enviar mensagens}
  
  Este caso de uso descreve a funcionalidade de iniciar uma troca de
  mensagens com o proponente, encerrar a troca de mensagens e visualizar histórico de mensagens do projeto. 
  
  \subsubsection{Transações}

  \begin{table*}[!h]
  \centering
  \caption{Transações do UC15}
  \label{uc15_transactions}
    \begin{tabular}{|p{0.20\linewidth}|p{0.25\linewidth}|p{0.20\linewidth}|}
    \hline
    \textbf{Transação} & \textbf{Fluxo} & \textbf{Passos} \\ 
    \hline
    1 & Básico & P1\\
    \hline
    2 & Básico & P2 e P3\\
    \hline
    3 & Básico & P4 e P5\\
    \hline
    4 & Alternativo & A2 e A3\\
    \hline
    5 & Alternativo & A6 e A7\\
    \hline
    6 & Alternativo & A10 e A11\\
    \hline
    7 & Alternativo & A12 e A13\\
    \hline
    8 & Alternativo & A16 e A17\\
    \hline
    \end{tabular}
  \end{table*}

  \subsubsection{Pontos de Caso de Uso}

  \begin{itemize}
  \item Quantidade de transações: 8;
  \item Complexidade: Alta;
  \item Total: 15.
  \end{itemize}

  \vfill
  
\pagebreak
\subsection{UC16 - Submeter Relatório Parcial e Final}
  
  Este caso de uso especifica a funcionalidade de submeter relatório parcial e final referente à execução do projeto.
  O usuário deverá ser capaz de fazer upload dos arquivos solicitados para cadastrar os relatórios, visualizar os
  relatórios cadastrados e submete-los para análise.
  
  \subsubsection{Transações}

  \begin{table*}[!h]
  \centering
  \caption{Transações do UC16}
  \label{uc16_transactions}
    \begin{tabular}{|p{0.20\linewidth}|p{0.25\linewidth}|p{0.20\linewidth}|}
    \hline
    \textbf{Transação} & \textbf{Fluxo} & \textbf{Passos} \\ 
    \hline
    1 & Básico & Início do caso de uso e P1\\
    \hline
    2 & Básico & P2 e P3\\
    \hline
    3 & Básico & P3 e P4\\
    \hline
    4 & Alternativo & A1.1 e A1.2\\
    \hline
    5 & Alternativo & A1.3 e A1.4\\
    \hline
    6 & Alternativo & A1.5 e A1.6\\
    \hline
    7 & Alternativo & A2.1 e A2.2\\
    \hline
    8 & Alternativo & A3.1 e A3.2\\
    \hline
    9 & Alternativo & A3.3 e A3.4\\
    \hline
    10 & Alternativo & A4.1 e A4.2\\
    \hline
    11 & Alternativo & A4.3 e A4.4\\
    \hline
    12 & Alternativo & A5.1 e A5.2\\
    \hline
    13 & Alternativo & A6.1 e A6.2\\
    \hline
    14 & Alternativo & A6.3 e A6.4\\
    \hline
    15 & Alternativo & A7.1 e A7.2\\
    \hline
    16 & Alternativo & A8.1 e A8.2\\
    \hline
    \end{tabular}
  \end{table*}

  \subsubsection{Pontos de Caso de Uso}

  \begin{itemize}
  \item Quantidade de transações: 16;
  \item Complexidade: Alta;
  \item Total: 15.
  \end{itemize}

  \vfill

\pagebreak
\subsection{UC17 - Analisar Relatório}
  
  Este caso de uso especifica a funcionalidade de analisar relatório parcial e final referente à execução do projeto.
  O usuário deverá ser capaz de visualizar os dados dos relatórios cadastrados e avaliar os relatórios.
  
  \subsubsection{Transações}

  \begin{table*}[!h]
  \centering
  \caption{Transações do UC17}
  \label{uc17_transactions}
    \begin{tabular}{|p{0.20\linewidth}|p{0.25\linewidth}|p{0.20\linewidth}|}
    \hline
    \textbf{Transação} & \textbf{Fluxo} & \textbf{Passos} \\ 
    \hline
    1 & Básico & Início do caso de uso e P1\\
    \hline
    2 & Alternativo & A1.1 e A1.2\\
    \hline
    3 & Alternativo & A2.1 e A2.2\\
    \hline
    4 & Alternativo & A4.1 e A4.2\\
    \hline
    5 & Alternativo & A4.3 e A4.4\\
    \hline
    6 & Alternativo & A5.1 e A5.2\\
    \hline
    7 & Alternativo & A5.3 e A5.4\\
      \hline
    8 & Alternativo & A7.1 e A7.2\\
      \hline
    9 & Alternativo & A7.3 e A7.4\\
      \hline
    10 & Alternativo & A8.1 e A8.2\\
      \hline
    11 & Alternativo & A10.1 e A10.2\\
    \hline
    \end{tabular}
  \end{table*}

  \subsubsection{Pontos de Caso de Uso}

  \begin{itemize}
  \item Quantidade de transações: 12;
  \item Complexidade: Alta;
  \item Total: 15.
  \end{itemize}
  
  \subsubsection{Suposições e observações}
  
  \begin{itemize}
   \item Os passos A3.1 e A3.2 não foram contados como uma transação, pois já haviam sido contados no UC16;	
   \item os passos A6.1 e A6.2 não foram contados como uma transação, pois já haviam sido contados no UC16;	
   \item Os passos A9.1 e A9.2 não foram contados como uma transação, pois já haviam sido contados no UC16;
  \end{itemize}
  
  \vfill
  
  
  
  \pagebreak
\subsection{UC18 - Visializar Relatórios}
  
  Esse caso de uso especifica a funcionalidade de visualizar relatório parcial e final referente à execução do projeto.
  O usuário deverá ser capaz de visualizar os dados dos relatórios cadastrados e alterar o responsável pelo acompanhamento.

  \subsubsection{Transações}

  \begin{table*}[!h]
  \centering
  \caption{Transações do UC18}
  \label{uc17_transactions}
    \begin{tabular}{|p{0.20\linewidth}|p{0.25\linewidth}|p{0.20\linewidth}|}
    \hline
    \textbf{Transação} & \textbf{Fluxo} & \textbf{Passos} \\ 
    \hline
    1 & Básico & P1\\
    \hline
    2 & Alternativo & A26 e A27\\
    \hline
    3 & Alternativo & A28 e A29\\
    \hline
    \end{tabular}
  \end{table*}

  \subsubsection{Pontos de Caso de Uso}

  \begin{itemize}
  \item Quantidade de transações: 3;
  \item Complexidade: Simples;
  \item Total: 5.
  \end{itemize}
  
  \subsubsection{Suposições e observações}
  
  \begin{itemize}
   \item Os passos A2 e A3 não foram contados como uma transação, pois já haviam sido contados no UC17;	
   \item os passos A6 e A7 não foram contados como uma transação, pois já haviam sido contados no UC17;	
   \item Os passos A10 e A11 não foram contados como uma transação, pois já haviam sido contados no UC16	;
   \item Os passos A14 e A15 não foram contados como uma transação, pois já haviam sido contados no UC16;	
   \item Os passos A18 e A19 não foram contados como uma transação, pois já haviam sido contados no UC16;
   \item Os passos A22 e A23 não foram contados como uma transação, pois já haviam sido contados no UC17;
  \end{itemize}
  
  \vfill
  
  
  
\pagebreak
\subsection{UC19 - Extrair Dados}
  
  Esse caso de uso especifica a funcionalidade de gerar relatório com a extração da base de dados do REPNBL. 
  O usuário deverá ser capaz de visualizar os dados em planilha excel.

  \subsubsection{Transações}

  \begin{table*}[!h]
  \centering
  \caption{Transações do UC19}
  \label{uc17_transactions}
    \begin{tabular}{|p{0.20\linewidth}|p{0.25\linewidth}|p{0.20\linewidth}|}
    \hline
    \textbf{Transação} & \textbf{Fluxo} & \textbf{Passos} \\ 
    \hline
    1 & Básico & P1\\
    \hline
    2 & Alternativo & A2 e A3\\
    \hline
    3 & Alternativo & A6 e A7\\
    \hline
    \end{tabular}
  \end{table*}

  \subsubsection{Pontos de Caso de Uso}

  \begin{itemize}
  \item Quantidade de transações: 3;
  \item Complexidade: Simples;
  \item Total: 5.
  \end{itemize}
   
  \vfill
  
  
  
  
\pagebreak
\subsection{UC20 - Visualizar acompanhamento}
  
  Este caso de uso especifica a funcionalidade Relatório Gerencial, que permite gerar 
  relatórios de acompanhamentos dos projetos do sistema REPNBL.
  
  \subsubsection{Transações}

  \begin{table*}[!h]
  \centering
  \caption{Transações do UC20}
  \label{uc20_transactions}
    \begin{tabular}{|p{0.20\linewidth}|p{0.25\linewidth}|p{0.20\linewidth}|}
    \hline
    \textbf{Transação} & \textbf{Fluxo} & \textbf{Passos} \\ 
    \hline
    1 & Básico & P1\\
    \hline
    2 & Alternativo & A2 e A3 \\
    \hline
    3 & Alternativo & A4 e A5 \\
    \hline
    4 & Alternativo & A16 e A17\\
    \hline
    \end{tabular}
  \end{table*}

  \subsubsection{Pontos de Caso de Uso}

  \begin{itemize}
  \item Quantidade de transações: 4;
  \item Complexidade: Médio;
  \item Total: 10.
  \end{itemize}
  
  \subsubsection{Suposições e observações}
  
  \begin{itemize}   
   \item Os passos A8 e A9 não foram contados como uma transação, pois já haviam sido contados no fluxo alternativo
      A1 com os passos A2 e A3;
   \item Os passos A12 e A13 não foram contados como uma transação, pois já haviam sido contados no fluxo alternativo
      A1 com os passos A2 e A3;
  \end{itemize}

  \vfill
  
\pagebreak
\subsection{UC21 - Emitir relação de projetos}
  
  Este caso de uso especifica a funcionalidade de emitir relação de projetos, que permite gerar uma relação com
  os projetos localizados após uma pesquisa executada no sistema REPNBL.
  
  \subsubsection{Transações}

  \begin{table*}[!h]
  \centering
  \caption{Transações do UC21}
  \label{uc21_transactions}
    \begin{tabular}{|p{0.20\linewidth}|p{0.25\linewidth}|p{0.20\linewidth}|}
    \hline
    \textbf{Transação} & \textbf{Fluxo} & \textbf{Passos} \\ 
    \hline
    1 & Básico & P1\\
    \hline
    2 & Alternativo & A2 e A3 \\
    \hline
    \end{tabular}
  \end{table*}

  \subsubsection{Pontos de Caso de Uso}

  \begin{itemize}
  \item Quantidade de transações: 2;
  \item Complexidade: Baixa;
  \item Total: 5.
  \end{itemize}

  \vfill

\section{Valores e considerações para a escolha do nível dos fatores técnicos}

\section{Valores e considerações para a escolha do nível dos fatores ambientais}

\vfill
\pagebreak
\section{Resultado da contagem}

\begin{table*}[!h]
\centering
\caption{Pontos de Caso de Uso}
\label{Rotulo}
  \begin{tabular}{|p{0.20\linewidth}|p{0.25\linewidth}|p{0.20\linewidth}|p{0.20\linewidth}|}
  \hline
  \textbf{Atores} & \textbf{Casos de Uso} & \textbf{Total não ajustado} & \textbf{Total ajustado} \\ 
  \hline

  9 & & &\\
  \hline
  \end{tabular}
\end{table*}

\section{Suposições e observações gerais}

