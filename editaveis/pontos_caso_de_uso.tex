\subsection{Propósito e tipo da contagem}

\subsection{Escopo da contagem}

\subsection{Fontes de informação utilizadas para identificar os RFU usados na medição}
 
\subsection{Participantes, seus papéis e qualificações}

\subsection{Data}
25/09/2015

\subsection{Atores}

\begin{table*}[!h]
\centering
\caption{Atores do sistema}
\label{Rotulo}
  \begin{tabular}{|p{0.20\linewidth}|p{0.25\linewidth}|p{0.30\linewidth}|p{0.10\linewidth}|}
  \hline
  \textbf{Nome do ator} & \textbf{Complexidade} & \textbf{Justificativa} & \textbf{PCU} \\ 
  \hline

  Analista & Alta & Usuário humano que interage com o sistema através de interface gráfica & 3 \\
  \hline
  Proponente & Alta & Usuário humano que interage com o sistema através de interface gráfica & 3\\
  \hline
  Supervisor & Alta & Usuário humano que interage com o sistema através de interface gráfica & 3\\
  \hline
  \end{tabular}
\end{table*}


\subsection{Casos de Uso}

\subsubsection{UC14 - Ferramentas Administrativas}

\subsubsection{Transações}

\begin{table*}[!h]
\centering
\caption{Pontos de Caso de Uso}
\label{Rotulo}
  \begin{tabular}{|p{0.20\linewidth}|p{0.25\linewidth}|p{0.20\linewidth}|}
  \hline
  \textbf{Transação} & \textbf{Fluxo} & \textbf{Passos} \\ 
  \hline
  1 & Básico & P2 e P3\\
  \hline
  2 & Básico & P4 e P5\\
  \hline
  3 & Alternativo & A1 e A2\\
  \hline
  4 & Alternativo & A3 e A4\\
  \hline
  5 & Alternativo & A5 e A6\\
  \hline
  \end{tabular}
\end{table*}

\pagebreak
\subsubsection{Pontos de Caso de Uso}

\begin{itemize}
 \item Quantidade de transações: 5;
 \item Complexidade: Média;
 \item Total: 10.
\end{itemize}

\vfill

\subsubsection{UC15 - Enviar mensagens}

\subsubsection{Transações}

\begin{table*}[!h]
\centering
\caption{Pontos de Caso de Uso}
\label{Rotulo}
  \begin{tabular}{|p{0.20\linewidth}|p{0.25\linewidth}|p{0.20\linewidth}|}
  \hline
  \textbf{Transação} & \textbf{Fluxo} & \textbf{Passos} \\ 
  \hline
  1 & Básico & P2 e P3\\
  \hline
  2 & Básico & P4 e P5\\
  \hline
  3 & Alternativo & A2 e A3\\
  \hline
  4 & Alternativo & A6 e A7\\
  \hline
  5 & Alternativo & A10 e A11\\
  \hline
  6 & Alternativo & A12 e A13\\
  \hline
  7 & Alternativo & A16 e A17\\
  \hline
  \end{tabular}
\end{table*}

\pagebreak
\subsubsection{Pontos de Caso de Uso}

\begin{itemize}
 \item Quantidade de transações: 7;
 \item Complexidade: Média;
 \item Total: 10.
\end{itemize}

\subsubsection{UC16 - Submeter Relatório Parcial e Final}

\subsubsection{Transações}

\begin{table*}[!h]
\centering
\caption{Pontos de Caso de Uso}
\label{Rotulo}
  \begin{tabular}{|p{0.20\linewidth}|p{0.25\linewidth}|p{0.20\linewidth}|}
  \hline
  \textbf{Transação} & \textbf{Fluxo} & \textbf{Passos} \\ 
  \hline
  1 & Básico & P1\\
  \hline
  2 & Básico & P2 e P3\\
  \hline
  3 & Básico & P4 e P5\\
  \hline
  4 & Alternativo & A2 e A3\\
  \hline
  5 & Alternativo & A6 e A7\\
  \hline
  6 & Alternativo & A10 e A11\\
  \hline
  7 & Alternativo & A12 e A13\\
  \hline
  8 & Alternativo & A16 e A17\\
  \hline
  \end{tabular}
\end{table*}

\pagebreak
\subsubsection{Pontos de Caso de Uso}

\begin{itemize}
 \item Quantidade de transações: 8;
 \item Complexidade: Alta;
 \item Total: 15.
\end{itemize}

\vfill

\subsection{Valores e considerações para a escolha do nível dos fatores técnicos}

\subsection{Valores e considerações para a escolha do nível dos fatores ambientais}

\pagebreak
\subsection{Resultado da contagem}

\begin{table*}[!h]
\centering
\caption{Pontos de Caso de Uso}
\label{Rotulo}
  \begin{tabular}{|p{0.20\linewidth}|p{0.25\linewidth}|p{0.20\linewidth}|p{0.20\linewidth}|}
  \hline
  \textbf{Atores} & \textbf{Casos de Uso} & \textbf{Total não ajustado} & \textbf{Total ajustado} \\ 
  \hline

  9 & & &\\
  \hline
  \end{tabular}
\end{table*}


\subsection{Suposições e questões resolvidas}


