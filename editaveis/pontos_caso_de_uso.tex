\subsection{Propósito e tipo da contagem}

\subsection{Escopo da contagem}

\subsection{Fontes de informação utilizadas para identificar os RFU usados na medição}
As informações utilizadas para a contagem foram retiradas das especificações de caso de uso do sistema.

\subsection{Participantes, seus papéis e qualificações}

\subsection{Data}
25/09/2015

\subsection{Atores}

\begin{table*}[!h]
\centering
\caption{Atores do sistema}
\label{Rotulo}
  \begin{tabular}{|p{0.20\linewidth}|p{0.25\linewidth}|p{0.30\linewidth}|p{0.10\linewidth}|}
  \hline
  \textbf{Nome do ator} & \textbf{Complexidade} & \textbf{Justificativa} & \textbf{PCU} \\ 
  \hline

  Analista & Alta & Usuário humano que interage com o sistema através de interface gráfica & 3 \\
  \hline
  Proponente & Alta & Usuário humano que interage com o sistema através de interface gráfica & 3\\
  \hline
  Supervisor & Alta & Usuário humano que interage com o sistema através de interface gráfica & 3\\
  \hline
  \end{tabular}
\end{table*}


\subsection{Casos de Uso}

\subsubsection{UC14 - Ferramentas Administrativas}

\subsubsection{Transações}

\begin{table*}[!h]
\centering
\caption{Pontos de Caso de Uso}
\label{Rotulo}
  \begin{tabular}{|p{0.20\linewidth}|p{0.25\linewidth}|p{0.20\linewidth}|}
  \hline
  \textbf{Transação} & \textbf{Fluxo} & \textbf{Passos} \\ 
  \hline
  1 & Básico & P1\\
  \hline
  2 & Básico & P2 e P3\\
  \hline
  3 & Básico & P4 e P5\\
  \hline
  4 & Alternativo & A1 e A2\\
  \hline
  5 & Alternativo & A3 e A4\\
  \hline
  6 & Alternativo & A5 e A6\\
  \hline
  \end{tabular}
\end{table*}

\pagebreak
\subsubsection{Pontos de Caso de Uso}

\begin{itemize}
 \item Quantidade de transações: 6;
 \item Complexidade: Média;
 \item Total: 10.
\end{itemize}

\vfill

\subsubsection{UC15 - Enviar mensagens}

\subsubsection{Transações}

\begin{table*}[!h]
\centering
\caption{Pontos de Caso de Uso}
\label{Rotulo}
  \begin{tabular}{|p{0.20\linewidth}|p{0.25\linewidth}|p{0.20\linewidth}|}
  \hline
  \textbf{Transação} & \textbf{Fluxo} & \textbf{Passos} \\ 
  \hline
  1 & Básico & P1\\
  \hline
  2 & Básico & P2 e P3\\
  \hline
  3 & Básico & P4 e P5\\
  \hline
  4 & Alternativo & A2 e A3\\
  \hline
  5 & Alternativo & A6 e A7\\
  \hline
  6 & Alternativo & A10 e A11\\
  \hline
  7 & Alternativo & A12 e A13\\
  \hline
  8 & Alternativo & A16 e A17\\
  \hline
  \end{tabular}
\end{table*}

\pagebreak
\subsubsection{Pontos de Caso de Uso}

\begin{itemize}
 \item Quantidade de transações: 8;
 \item Complexidade: Alta;
 \item Total: 15.
\end{itemize}

\subsubsection{UC16 - Submeter Relatório Parcial e Final}

\subsubsection{Transações}

\begin{table*}[!h]
\centering
\caption{Pontos de Caso de Uso}
\label{Rotulo}
  \begin{tabular}{|p{0.20\linewidth}|p{0.25\linewidth}|p{0.20\linewidth}|}
  \hline
  \textbf{Transação} & \textbf{Fluxo} & \textbf{Passos} \\ 
  \hline
  1 & Básico & P1\\
  \hline
  2 & Básico & P2 e P3\\
  \hline
  3 & Básico & P3 e P4\\
  \hline
  4 & Alternativo & A2 e A3\\
  \hline
  5 & Alternativo & A4 e A5\\
  \hline
  6 & Alternativo & A6 e A7\\
  \hline
  7 & Alternativo & A10 e A11\\
  \hline
  8 & Alternativo & A14 e A15\\
  \hline
  9 & Alternativo & A16 e A17\\
  \hline
  10 & Alternativo & A20 e A21\\
  \hline
  11 & Alternativo & A22 e A23\\
  \hline
  12 & Alternativo & A26 e A27\\
   \hline
  13 & Alternativo & A30 e A31\\
  \hline
  14 & Alternativo & A32 e A33\\
   \hline
  15 & Alternativo & A36 e A37\\
  \hline
  16 & Alternativo & A40 e A41\\
  \hline
  \end{tabular}
\end{table*}

\pagebreak
\subsubsection{Pontos de Caso de Uso}

\begin{itemize}
 \item Quantidade de transações: 16;
 \item Complexidade: Alta;
 \item Total: 15.
\end{itemize}

\vfill

\subsubsection{UC17 - Analisar Relatório}

\subsubsection{Transações}

\begin{table*}[!h]
\centering
\caption{Pontos de Caso de Uso}
\label{Rotulo}
  \begin{tabular}{|p{0.20\linewidth}|p{0.25\linewidth}|p{0.20\linewidth}|}
  \hline
  \textbf{Transação} & \textbf{Fluxo} & \textbf{Passos} \\ 
  \hline
  1 & Básico & P1\\
  \hline
  2 & Alternativo & A3 e A4\\
  \hline
  3 & Alternativo & A6 e A7\\
  \hline
  4 & Alternativo & A13 e A14\\
  \hline
  5 & Alternativo & A15 e A16\\
  \hline
  6 & Alternativo & A19 e A20\\
   \hline
  7 & Alternativo & A21 e A22\\
   \hline
  8 & Alternativo & A29 e A30\\
     \hline
  9 & Alternativo & A31 e A32\\
     \hline
  10 & Alternativo & A35 e A36\\
     \hline
  11 & Alternativo & A37 e A38\\
     \hline
  12 & Alternativo & A45 e A46\\
  \hline
  \end{tabular}
\end{table*}

\pagebreak
\subsubsection{Pontos de Caso de Uso}

\begin{itemize}
 \item Quantidade de transações: 12;
 \item Complexidade: Alta;
 \item Total: 15.
\end{itemize}

\vfill

\subsection{Valores e considerações para a escolha do nível dos fatores técnicos}

\subsection{Valores e considerações para a escolha do nível dos fatores ambientais}

\pagebreak
\subsection{Resultado da contagem}

\begin{table*}[!h]
\centering
\caption{Pontos de Caso de Uso}
\label{Rotulo}
  \begin{tabular}{|p{0.20\linewidth}|p{0.25\linewidth}|p{0.20\linewidth}|p{0.20\linewidth}|}
  \hline
  \textbf{Atores} & \textbf{Casos de Uso} & \textbf{Total não ajustado} & \textbf{Total ajustado} \\ 
  \hline

  9 & & &\\
  \hline
  \end{tabular}
\end{table*}


\subsection{Suposições e observações}
  
  \begin{itemize}
   \item No UC17 o A9 e A10 não foram contados como uma transação, pois já haviam sido contados no UC16;	
   \item No UC17 o A25 e A26 não foram contados como uma transação, pois já haviam sido contados no UC16;	
   \item No UC17 o A41 e A42 não foram contados como uma transação, pois já haviam sido contados no UC16;	
\end{itemize}

  

