\chapter{Conclusão}

  Esta seção apresenta as conclusões feitas com a realização das contagens, utilizando as diferentes técnicas de medição e as estimativas de esforço
  para cada técnica.
  
  

  \section{Técnica de Pontos de Casos de Uso}
      
    A contagem por Pontos de Caso de Uso depende bastante da qualidade e do tipo da documentação disponível.
    No caso do sistema contado neste trabalho, a qualidade da documentação era bem precária, o que pode 
    ter influenciado negativamente a contagem, que pode apresentar relevante inconsistência. Em alguns casos, é difícil 
    identificar as transações inerentes a um caso de uso, o que dificulta a contagem.
    
    \subsection{Estimativa de Prazo}
   
      Como não houve nenhum fator ambiental (F1 a F6) menor que 3, o valor homems/hora considerado foi 20 (KARNER, 1993 \textit{apud} \cite{artigo_pcu}). Portanto, supondo que a equipe de desenvolvimento
      8 horas por dia, a estimativa de esforço é 1020 homens/hora, sendo necessário 128 dias para a conclusão da implementação dos casos de uso
      do escopo de contagem.
	
  \section{Técnica de Pontos de Função}
    
    A contagem por Pontos de Função é uma técnica mais elaborada que exige um esforço de análise muito maior em relação às outras duas 
    técnicas, por precisar identificar a quantidade de atributos de cada grupo de dados, sendo, às vezes, necessário uma documentação maior e mais técnica
    para o levantamento preciso de tais informações. A rigorosidade maior desta técnica a torna bem mais detalhada.
    
    Um fator que desvaloriza um pouco a contagem por pontos de função é que esta não conta a visualização de um mesmo 
    conjunto de dados simplesmente por estar em outro formato de apresentação. Por mais que seja o mesmo conjunto de dados, 
    a forma de apresentação impacta bastante tanto na implementação quanto na visão do usuário. Por exemplo, visualizar um relatório 
    em uma tabela na tela e gerar um arquivo PDF com o mesmo relatório, são funcionalidades diferentes, em questão de implementação, 
    e, sob o ponto de vista do usuário, o relatório em PDF agrega um valor diferente, o que devia ser considerado pela técnica.
    
  \section{Técnica de Pontos de Função COSMIC}
  
    A contagem por Pontos de Função COSMIC fornece um método mais simples e intuitivo de contar o tamanho funcional do software, 
    a partir das movimentações de dados realizadas interna e externamente. Além de ser mais fácil realizar a contagem, o resultado obtido 
    representa mais intuitivamente o tamanho do software, apresentando uma quantidade aproximada das transações realizadas por este sistema.
    
    
    \section{Resumo dos resultados}
\begin{table}[!h]
\centering
\caption{Pontos positivos e negativos de cada técnica}
\label{resumo_resultados}
\begin{tabular}{llll}
\hline
\multicolumn{1}{|c|}{\textbf{Técnica}}                                                  & \multicolumn{1}{c|}{\textbf{Valor}} & \multicolumn{1}{c|}{\textbf{Pontos positivos}}                                                                                                        & \multicolumn{1}{c|}{\textbf{Pontos negativos}}                                                                                                                                                        \\ \hline
\multicolumn{1}{|l|}{\begin{tabular}[c]{@{}l@{}}Pontos de\\ Caso de Uso\end{tabular}}   & \multicolumn{1}{l|}{51}             & \multicolumn{1}{l|}{\begin{tabular}[c]{@{}l@{}}- Mais simples de contar que PF\\ - Leva em conta os fatores \\    técnicos e ambientais\end{tabular}} & \multicolumn{1}{l|}{\begin{tabular}[c]{@{}l@{}}- Depende de um paradigma\\ - Depende da qualidade\\   da documentação\end{tabular}}                                                                   \\ \hline
\multicolumn{1}{|l|}{\begin{tabular}[c]{@{}l@{}}Pontos de \\ Função\end{tabular}}       & \multicolumn{1}{l|}{132}            & \multicolumn{1}{l|}{\begin{tabular}[c]{@{}l@{}}- Mundialmente difundido\\ - Não depende de um paradigma\end{tabular}}                                 & \multicolumn{1}{l|}{\begin{tabular}[c]{@{}l@{}}- Leva um tempo maior para contar\\ - Não leva em consideração\\    a visão técnica\\ - Não tem uma boa representação\\   na visão micro\end{tabular}} \\ \hline
\multicolumn{1}{|l|}{\begin{tabular}[c]{@{}l@{}}Pontos de\\ Função COSMIC\end{tabular}} & \multicolumn{1}{l|}{107}            & \multicolumn{1}{l|}{\begin{tabular}[c]{@{}l@{}}- Mais simples de contar\\ - Tem uma boa representação\\    na visão micro\end{tabular}}               & \multicolumn{1}{l|}{\begin{tabular}[c]{@{}l@{}}- Não leva em consideração\\    a visão técnica\end{tabular}}                                                                                          \\ \hline
                                                                                        &                                     &                                                                                                                                                       &                                                                                                                                                                                                       \\
                                                                                        &                                     &                                                                                                                                                       &                                                                                                                                                                                                       \\
                                                                                        &                                     &                                                                                                                                                       &                                                                                                                                                                                                      
\end{tabular}
\end{table}
	