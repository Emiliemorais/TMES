\chapter{Apresentação do Sistema}

O sistema contado é o REPNBL – Fase 2 – Sistema de Apresentação de Projetos ao Programa Nacional de Banda Larga,
que é utilizado pelo Ministério das Comunicações.

\section{Domínio do Software}

\section{Objetivos}

\section{Usuários do Software}

\section{Requisitos}

\section{Contagem}

    \subsection{Propósito da contagem}
	    Medir o tamanho 

    \subsection{Escopo da contagem}
      
      O escopo da contagem definido é composto pelos 8 casos de uso seguintes:
	    
      \begin{itemize}
	      

	  \item UC14 - Ferramentas administrativas;
	  \item UC15 - Enviar mensagens;
	  \item UC16 - Submeter relatório parcial e final;
	  \item UC17 - Analisar relatório;
	  \item UC18 - Visualizar relatório;
	  \item UC19 - Extrair dados;
	  \item UC20 - Visualizar acompanhamento;
	  \item UC21 - Emitir relação de projetos;	

      \end{itemize}
      
      Este mesmo escopo de contagem foi utilizado nas três abordagens de contagem: Pontos de Casos de Uso, Pontos de Função e Pontos de Função COSMIC.
      
      \subsubsection{Aspectos das funcionalidades medidas}
	  
	  As funcionalidades do software medido representam um software pronto que está em produção no Ministério das Comunicações,
	  caracterizando as funcionalidades como entregues.
      
    \subsection{Documentação utilizada}
	
	Os seguintes artefatos foram utilizados como fonte de informações para identificar os requisitos funcionais do \textit{software}.
	
	\begin{itemize}
	  \item Especificações de Caso de Uso;
	  \item Regras de negócio;
	  \item Proposta de Especificação do Software;
	  \item Protótipos de telas;
	\end{itemize}
	
    \subsection{Participantes, seus papéis e qualificações}
